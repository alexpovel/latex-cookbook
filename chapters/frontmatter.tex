% Helper file that pulls subchapters together.

% In conjunction, packages polyglossia and cleveref do some serious witchcraft.
% If we call these \crefname redefinitions in the preamble, it does not work.
% Also does not work with \AtBeginDocument. Tried with both a macro inside the
% redefinition, and just plain text.
\crefname{listing}{\lstlistingname}{\lstlistingname s}
\Crefname{listing}{\lstlistingname}{\lstlistingname s}

% Set initial pages to alpha, so that they do not collide with later Arabic numbering
% in the generated PDF. This won't show in print because page numbers aren't displayed
% until later. But you will be able to print the title page by printing page 'a', which
% would otherwise overlap with page '1', aka the first actual text page.
\pagenumbering{alph}
\maketitle% Print main title page
\subimport{frontmatter/}{colophon}

\makeatletter
    % Switch titlepage style around
    \ifdefstring{\cookbook@titlestyle}{book}{%
        \renewcommand*{\cookbook@titlestyle}{thesis}%
    }{%
        \renewcommand*{\cookbook@titlestyle}{book}%
    }
    \subtitle{An alternative title page style}
    %
    % Run this file again, since it contains the \renewcommand definitions
    
% Emulating a budget 'case' switch statement here

% \ifstrequal from etoolbox doesn't work, it does not expand macros.
% Use \ifdefstring from same package,
% see also https://tex.stackexchange.com/a/451937/120853

\ifdefstring{\cookbook@titlestyle}{thesis}{
    % These exist already, overwrite:
    \setkomafont{title}{\sffamily\bfseries\Huge}
    \setkomafont{subtitle}{\normalfont\normalsize\sffamily}
    \setkomafont{author}{\sffamily\bfseries}
    \setkomafont{publishers}{\sffamily}
%
    % One same line as subtitle, so should look the same; inherit style:
    \setkomafont{date}{\usekomafont{subtitle}}
%
    \setkomafont{documenttype}{\usekomafont{subtitle}\footnotesize}
%
    \renewcommand{\maketitle}{%
        \begin{titlepage}
            \hfill%
            %
            % 2nd optional argument is height, required for \vfill etc. to work
            \begin{minipage}[b][1\textheight]{0.05\textwidth}%
                \color{black}\rule{0.15em}{\textheight}% Vertical line
                \hfill%
            \end{minipage}% This comment char is important to suppress linebreak
            %
            \begin{minipage}[b][1\textheight]{0.75\textwidth}
                \vspace{4\baselineskip}
                {\usekomafont{documenttype}\faGraduationCap{}\ \@documenttype{}\hspace{0.5em}}
                \hrulefill
                \begin{spacing}{1.1}% In case of multi-line titles, more relaxed spacing
                    \usekomafont{title}\raggedright\@title
                \end{spacing}
                \hrule
                \vspace{0.8\baselineskip}
                {%
                    \usekomafont{author}%
                    % Tabular enables \and and \\ syntax
                    \begin{tabular}[t]{@{}l@{}}%
                        \@author
                    \end{tabular}
                }%
                \hfill
                {\usekomafont{date}\@date}\par
                \vfill
                {\usekomafont{subtitle}\@subtitle}\par
                \vspace{2\baselineskip}
                {%
                    \usekomafont{publishers}%
                    \begin{tabular}[t]{@{}l@{}}%
                        \@publishers
                    \end{tabular}
                }\par
                \vspace{0.2ex}% Also required for \vfill as something to 'fill against'
            \end{minipage}
        \end{titlepage}
    }%
}{%
    % Else: nothing; faking a 'case' environment
}

\ifdefstring{\cookbook@titlestyle}{book}{
    % These exist already, overwrite:
    \setkomafont{title}{\sffamily\bfseries\Huge}
    \setkomafont{subtitle}{\itshape}
    \setkomafont{author}{\scshape\Large}
    \setkomafont{date}{\normalfont\normalsize\sffamily}
    \setkomafont{publishers}{\scshape}
%
    % One same line as date, so should look the same; inherit style:
    \setkomafont{documenttype}{\usekomafont{date}}
%
    \renewcommand{\maketitle}{%
        \begin{titlepage}
            \hfill%
            %
            % 2nd optional argument is height, required for \vfill etc. to work
            \begin{minipage}[b][1\textheight]{0.05\textwidth}%
                \color{black}\rule{0.15em}{\textheight}% Vertical line
                \hfill%
            \end{minipage}% This comment char is important to suppress linebreak
            %
            \begin{minipage}[b][1\textheight]{0.75\textwidth}
                \vspace{2\baselineskip}
                {\usekomafont{author}
                    % Tabular enables \and and \\ syntax
                    \begin{tabular}[t]{@{}l@{}}%
                        \@author
                    \end{tabular}
                }\par
                \vspace{1.5\baselineskip}
                \begin{spacing}{1.1}% In case of multi-line titles, more relaxed spacing
                    \usekomafont{title}\raggedright\@title
                \end{spacing}
                \hrule
                \vspace{0.8\baselineskip}
                {\usekomafont{documenttype}\@subtitle}
                \hfill
                {\usekomafont{date}\@date}\par
                \vfill
                {\usekomafont{publishers}
                \begin{tabular}[t]{@{}l@{}}%
                    \@publishers
                    \end{tabular}
                }\par
                \vspace{0.2ex}% Also required for \vfill as something to 'fill against'
            \end{minipage}
        \end{titlepage}
    }%
}{
    % Else: nothing; faking a 'case' environment
}%

\makeatother

\maketitle% Print alternative title page

\frontmatter% In KOMAScript, resets pagenumber, uses Roman numerals etc.

% Note that \subincludefrom{}{} cannot be nested, therefore us \subimport
\subimport{frontmatter/}{task}
\subimport{frontmatter/}{authorship_declaration}
\subimport{frontmatter/}{abstract}

%%%%%%%%%%%%%%%%%%%%%%%%%%%%%%%%%%%%%%%%%%%%%%%%%%%%%%%%
% Lists of Content
%%%%%%%%%%%%%%%%%%%%%%%%%%%%%%%%%%%%%%%%%%%%%%%%%%%%%%%%

\tableofcontents

%%%%%%%%%%%%%%%%%%%%%%%%%%%%%%%%%%%%%%%%%%%%%%%%%%%%%%%%
% addchap is KOMA equivalent for \chapter*, but also creates ToC entry, see also
% https://tex.stackexchange.com/a/116085/120853
% Use built-in macro \glossaryname for proper internationalization. With polyglossia, it
% will contain \text<language>{<glossary translation>}, which has been taken care of
% using \pdfstringdefDisableCommands{} in the class file
\addchap{\glossaryname}%
\label{ch:glossary}

\emph{%
    \TransGlossaryLegend{}%
}%

% Print "unsorted" glossaries; these are in fact sorted, but externally using bib2gls.
% These will throw 'Token not allowed in PDF, removing \text<language>' warning.
% Specify title= manually if that gets too annoying.
\printunsrtglossary[
    type=symbols,
    style=symbunitlong,
]
\printunsrtglossary[
    type=numbers,
    style=numberlong,
]
\printunsrtglossary[
    type=subscripts,
    style=mcolalttree,
    nonumberlist,
]
\printunsrtglossary[
    type=abbreviations,
    style=long3colheader,
]

%%%%%%%%%%%%%%%%%%%%%%%%%%%%%%%%%%%%%%%%%%%%%%%%%%%%%%%%
\listoffigures%

\listoftables%

\listofexamples%

\lstlistoflistings%

\subimport{frontmatter/}{preface}
