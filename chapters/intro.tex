\chapter{Thesis Features}

This template is the result of a Master thesis in \textbf{mechanical engineering}.
Keeping that in mind, it is probably still absolutely suitable for countless other disciplines.
Adapt and use this template as you wish; giving credit to the original repository at
\begin{center}
	\url{https://github.com/alexpovel/thesis_template}
\end{center}
would be appreciated, so that collaborators can be attracted.

\section{Text Features}
Between any section commands (from \verb|\part| down), there always has to be some content.
I would normally not know what to put between \textbf{Text Features} and \textbf{Fonts}, but really putting \textit{anything} I consider necessary.
Anything else looks off.
At least, tell the reader what the following hierarchical level contains for them.
\subsection{Fonts}
Using high-quality fonts is one goal.
This includes the fantastic \href{https://ctan.org/texarchive/fonts/tex-gyre/opentype}{\TeX Gyre fonts}, of which the \textit{Palatino} (by Hermann Zapf) clone \textit{Pagella} was chosen for this template.
It comes with an accompanying \href{https://ctan.org/texarchive/fonts/tex-gyre-math/opentype}{math font} of the same name.
As such, achieving a better match of text and maths fonts is nigh impossible.
Not only do the fonts look fantastic on their own, they (just as important) also feature an immensely broad support for symbols and characters, as well as font shapes and weights and combinations thereof.
The latter is demonstrated in \cref{tab:font_examples}.
Note that using the package \texttt{cleveref}, we only ever issue \verb|\cref| commands.
The package does the heavy lifting and puts the float type in front, also with correct plural forms if required.
Doing it any other way is just way too laborious.
%
\newcommand*{\sampletext}{The quick brown Fox jumps over the lazy Dog 13 times!}
\begin{table}
\ttabbox{%
	\caption[\TeX Gyre Pagella Examples]% Brackets hold entry that will go into List of Tables
	{%
		Examples for font features offered by \TeX Gyre Pagella%
	}%
	\label{tab:font_examples}% Use a string like 'tab:' to help with organization and auto-complete when reffing
}%
{%
	\begin{tabular}{@{}ll@{}}% Use @{} to remove white space from sides (since braces are empty)
		\toprule
			Feature & Sample Text\\
		\midrule
			Regular & \sampletext\\
			\textbf{Bold} & \textbf{\sampletext}\\
			\textit{Italics} & \textit{\sampletext}\\
			\textbf{\textit{Bold Italics}} & \textbf{\textit{\sampletext}}\\
		\addlinespace% Some ininconspicuous vertical separation; more visually pleasing than a full rule
			\textsc{Small Capitals} & \textsc{\sampletext}\\
			\textbf{\textsc{Bold SC}} & \textbf{\textsc{\sampletext}}\\
			\textit{\textsc{Italics SC}} & \textit{\textsc{\sampletext}}\\
		\bottomrule
	\end{tabular}
}%
\end{table}
%%%%%%%%%%%%%%%%%%%%%%%%%%%%%%%%%%%%%%%%%%%%%%%%%%%%%%%%%%%%%%%%%%%%%%%%%%%%%%%%%%%%%%%%%%%
\subsection{Lists}
In technical publications, using lists (either bullet points or enumerations) is highly encouraged.
They should \textbf{always} be preferred over doing the same thing in a block of text.
Just consider this example list:
\begin{itemize}
	\item the presented approach is more complex than the previous one:
	\begin{enumerate}
		\item more time was spent doing computation,
		\item less time was spent effing about,
		\item features were added.
	\end{enumerate}
	\item at the same time, the following simplifications were made:
	\begin{enumerate}
		\item went from continuous- to discrete-time simulations,
		\item threw out some superfluous stuff.
	\end{enumerate}
\end{itemize}
Processing the very same information from a text paragraph is much less accessible.
Note the block-like \texttt{itemize} symbol, and the fact that enumerate numbers are part of the sans-serif family and bold.
This was specified and may be changed in the \texttt{enumitem} package options.
%%%%%%%%%%%%%%%%%%%%%%%%%%%%%%%%%%%%%%%%%%%%%%%%%%%%%%%%%%%%%%%%%%%%%%%%%%%%%%%%%%%%%%%%%%%
\subsection{Censoring}
Let me tell you a huge secret: \todo{I am a TODO note} \censor{today is \today}.
You can also censor float contents, as illustrated in \cref{fig:censorbox}.
\begin{figure}
	\ffigbox[\FBwidth]
	{%
		\caption{Example for a censoring box}%
		\label{fig:censorbox}%
	}%
	{%
		\censorbox{%
		\includegraphics[width=0.5\textwidth]{example_image}% Having issued \graphicspath globally, we do not have to specify the full path here. Not even the file extension is strictly necessary.
		}%
	}
\end{figure}
%%%%%%%%%%%%%%%%%%%%%%%%%%%%%%%%%%%%%%%%%%%%%%%%%%%%%%%%%%%%%%%%%%%%%%%%%%%%%%%%%%%%%%%%%%%
%%%%%%%%%%%%%%%%%%%%%%%%%%%%%%%%%%%%%%%%%%%%%%%%%%%%%%%%%%%%%%%%%%%%%%%%%%%%%%%%%%%%%%%%%%%
\section{Floats: Figures, Tables etc.}
In \cref{fig:censorbox,tab:font_examples}, we have already seen examples for floats.
These are handled by packages \texttt{caption} and \texttt{floatrow}.
Notable features are the capability to have captions the same width as the float they are attached to.
This tends to look much prettier, \iecfeg{cf.} \cref{fig:wide_caption,fig:tighter_caption}.
%%%%%%%%%%%%%%%%%%%%%%%%%%%%%%%%%%%%%%%%%%%%%%%%%%%%%%%%%%%%%%%%%%%%%%%%%%%%%%%%%%%%%%%%%%%
\begin{figure}
\includegraphics[width=0.4\textwidth]{example_image}%
\caption{Example for a regular caption, spanning the whole width since it is so long}%
\label{fig:wide_caption}%
\end{figure}
\begin{figure}
\ffigbox[\FBwidth]% Optional argument FBwidth causes graphic to be its actual size and caption to take up the remaining space. Otherwise, horizontal space is split 50/50, which doesn't make much sense.
{%
	\caption{Example for a new caption, spanning the just the width of the float it is attached to}%
	\label{fig:tighter_caption}%
}%
{%
	\includegraphics[width=0.4\textwidth]{example_image}%
}%
\end{figure}
%%%%%%%%%%%%%%%%%%%%%%%%%%%%%%%%%%%%%%%%%%%%%%%%%%%%%%%%%%%%%%%%%%%%%%%%%%%%%%%%%%%%%%%%%%%
\paragraph{Multiple floats}
Other possibilities are rather arbitrary arrangements of sub-figures and -captions.
For this, see \cref{fig:impeller_throat}, which contains two sub-figures \cref{fig:impeller_throat_iso,fig:impeller_throat_flat}.
Multiple sub-tables are also possible, see \cref{tab:works_on_turbocharging}.
\begin{figure}
\ffigbox[\FBwidth]
{%
	\begin{subfloatrow}[2]% Default is two anyway
		\ffigbox[\FBwidth]
		{%
			\def\svgwidth{0.4\figurewidth}
			\input{./images/impeller_isometric.pdf_tex}% input is not includegraphics, so we have to give the full path
		}%
		{%
			\caption{Isometric}%
			\label{fig:impeller_throat_iso}%
		}%
		\ffigbox[\FBwidth]
		{%
			\def\svgwidth{0.3\figurewidth}
			\input{./images/inducer_throat.pdf_tex}
		}%
		{%
			\caption{Schematic}%
			\label{fig:impeller_throat_flat}%
		}%
	\end{subfloatrow}
}%
{%
	\caption[Impeller Throat]%
	{%
		Inducer throat created by geometric constraints%
	}%
	\label{fig:impeller_throat}%
	\floatfoot{\adaptedfrom{} \autocites{shakal_centrifugal_2015}[99]{whitfield_design_1990}[535]{hayami_flow_1985}}%
}%
\end{figure}
\begin{table}
	\tiny
	\floatbox{table}%
	{%https://tex.stackexchange.com/q/295567/120853
		\begin{subfloatrow}
			\ttabbox[\FBwidth]%
			{%
				\begin{tabular}{@{}m{0.25\textwidth}m{0.15\textwidth}@{}}
					\toprule
					Author \& [Work] & Comment\\
					\midrule
					\textcite{bozza_map-based_2011} & ---\\
					\textcite{bozza_numerical_2013} & ---\\
					\textcite{bozza_theoretical_1990} & \textbf{Nozzles}\\
					\textcite{burke_modelling_2014} & ---\\
					\textcite{de_bellis_development_2018} & ---\\
					\textcite{berndt_einfluss_2009} & ---\\
					\textcite{chesse_performance_2000} & ---\\
					\textcite{cornolti_1d_2013} & ---\\
					\textcite{eriksson_modeling_2007} & ---\\
					\textcite{grigoriadis_experimentelle_2008} & ---\\
					\textcite{kech_model-based_2002} & ---\\
					\textcite{lee_simulation-based_2009} & ---\\
					\textcite{leufven_surge_2011} & ---\\
					\textcite{shaaban_part-load_2006} & ---\\
					\textcite{wahlstrom_modelling_2011} & ---\\
					\textcite{watel_matching_2010} & \mtlbsmlnk{}\\
					\textcite{yang_mixed_2010} & ---\\
					\bottomrule
				\end{tabular}
			}%
			{%
				\caption{Using Maps}%
				\label{tab:works_using_maps}%
			}%
			\ttabbox[\FBwidth]%
			{%
				\begin{tabular}{@{}m{0.25\textwidth}m{0.15\textwidth}@{}}
					\toprule
					Author \& [Work] & Comment\\
					\midrule
					\textcite{beckey_compressor_2011} & using algorithms\\
					\textcite{bergqvist_prediction_2014} & from CFD\\
					\textcite{bohn_modellierung_2002} & ---\\
					\textcite{bolz_critical_2014} & ---\\
					\textcite{elkamel_experimental_2011} & ---\\
					\textcite{ewert_modellierung_2013} & ---\\
					\textcite{freeman_compressor_2011} & \mtlb{} script\\
					\textcite{hansch_untersuchung_2010} & ---\\
					\textcite{harley_inlet_2014} & ---\\
					\textcite{luddecke_engine_2014} & ---\\
					\textcite{kurzke_correlations_2011} & ---\\
					\textcite{schwarz_considerations_2014} & ---\\
					\bottomrule
				\end{tabular}
			}%
			{%
				\caption{Computing Maps}%
				\label{tab:works_computing_maps}%
			}%
		\end{subfloatrow}
		
		\begin{subfloatrow}
			\ttabbox[\FBwidth]%
			{%
				\begin{tabular}{@{}m{0.25\textwidth}m{0.15\textwidth}@{}}
					\toprule
					Author \& [Work] & Comment\\
					\midrule
					\textcite{aungier_mean_1995} & ---\\
					\textcite{casey_method_2013} & Calibrating with Maps\\
					\textcite{chen_one-dimensional_2014} & Turbine\\
					\textcite{erickson_centrifugal_2008} & \fortran{}\\
					\textcite{ewert_modellierung_2013} & ---\\
					\textcite{gong_total_2014} & ---\\
					\textcite{gutierrez_velasquez_one_2010} & \fortran{}, 1D and 3D\\
					\textcite{japikse_turbomachinery_2009} & Overview\\
					\textcite{kamaleshaiah_improved_1988} & ---\\
					\textcite{kessel_modellbildung_2002} & 1- \& 3D\\
					\textcite{lee_dual-stage_2008} & dual-stage; \textbf{\smlnk{}, maps}\\
					\textcite{nakhjiri_vatl_2018} & ---\\
					\textcite{okhuahesogie_1-d_2014} & Differential Evolution\\
					\textcite{pelton_one-dimensional_2007} & ---\\
					\textcite{pixberg_modellbildung_2013} & \textbf{Reservoirs; Nozzles} (p.\ 42)\\
					\textcite{sakellaridis_development_2015} & ---\\
					\textcite{schiffmann_design_2010} & ---\\
					\textcite{schlador_erstellung_2008} & \fortran{}, \textbf{Reservoirs; Models}\\
					\textcite{schur_transient_2013} & \smlnk{}\\
					\textcite{serrano_model_2008} & \textbf{Reservoirs; Nozzles}\\
					\textcite{shaaban_experimental_2004} & Diabatic; Extrapolation\\
					\textcite{stuart_analysis_2014} & cites \autocite{harley_evaluation_2013}\\
					\textcite{taburri_model-based_2012} & \textbf{Reservoirs; Nozzles}\\
					\textcite{uchida_transient_2006} & ---\\
					\textcite{whitfield_preliminary_1990} & Preliminary Design\\
					\textcite{wirz_berechnungsmethode_2017} & \textbf{Reservoirs; Models}\\
					\textcite{xu_empirical_2012} & ---\\
					\textcite{zahn_arbeitsspielaufgeloste_2012} & ---\\
					\textcite{zhuge_development_2009} & ---\\
					\bottomrule
				\end{tabular}
			}%
			{%
				\caption{Physical (0- \& 1D)}%
				\label{tab:works_physical}%
			}%
			\ttabbox[\FBwidth]%
			{%
				\begin{tabular}{@{}m{0.25\textwidth}m{0.15\textwidth}@{}}
					\toprule
					Author \& [Work] & Comment\\
					\midrule
					\textcite{aungier_mean_1995} & ---\\
					\textcite{doustmohammadi_experimental_2013} & ---\\
					\textcite{el-maksoud_prediction_2012} & ---\\
					\textcites{galvas_analytical_1972}{galvas_fortran_1973} & ---\\
					\textcite{gulich_disk_2003} & Disk Friction\\
					\textcite{gutierrez_velasquez_determination_2017} & ---\\
					\textcite{harley_evaluation_2013} & Recommends \autocite{oh_optimum_1997} \\
					\textcite{mohtar_increasing_2010} & ---\\
					\textcite{nakhjiri_physical_2011} & ---\\
					\textcite{oh_optimum_1997} & ---\\
					\textcite{tacconi_investigation_2018} & Cites \autocite{harley_evaluation_2013}; recommends \autocite{galvas_fortran_1973}\\
					\textcite{schneider_analytical_2015} & 2D\\
					\textcite{solaesa_analytical_2016} & ---\\
					\bottomrule
				\end{tabular}
			}%
			{%
				\caption{Loss Modelling}%
				\label{tab:works_losses}%
			}%
		\end{subfloatrow}%
	}%
	{%
		\caption[Works on zero- and one-dimensional modelling]
		{%
			Works on zero- and one-dimensional turbocharger and engine modelling%
		}%
		\label{tab:works_on_turbocharging}%
	}%
\end{table}
%%%%%%%%%%%%%%%%%%%%%%%%%%%%%%%%%%%%%%%%%%%%%%%%%%%%%%%%%%%%%%%%%%%%%%%%%%%%%%%%%%%%%%%%%%%
\paragraph{Side Captions}
Lastly, on occasion figures and their captions might look disproportionate in combination.
In these cases, placing a side-caption might relieve the situation, as shown in \cref{fig:sidecap}.
\begin{figure}
\fcapside[\FBwidth]%
{%
	\caption{A side caption, which may also span multiple lines like demonstrated in this rather long caption right here}
	\label{fig:sidecap}%
}%
{%
	\includegraphics[width=0.4\textwidth]{example_image}%
}%
\end{figure}
%%%%%%%%%%%%%%%%%%%%%%%%%%%%%%%%%%%%%%%%%%%%%%%%%%%%%%%%%%%%%%%%%%%%%%%%%%%%%%%%%%%%%%%%%%%
\paragraph{Caption Position}
Note that the caption of \cref{tab:font_examples} occurs \textit{above} the table no matter the \texttt{caption} command's position.
Per convention, figure captions should appear below, table captions above their bodies.
This is also handled by \texttt{floatrow}.
Also note that there is neither a fullstop nor \textit{any} character (no space, no empty line) behind the last caption line in the source code, since dots are managed globally by the \texttt{caption} package.
\paragraph{Float Footer}
We also have a \verb|\floatfoot| command for all floats.
This is used to place additional info underneath the caption, primarily used for references.
%%%%%%%%%%%%%%%%%%%%%%%%%%%%%%%%%%%%%%%%%%%%%%%%%%%%%%%%%%%%%%%%%%%%%%%%%%%%%%%%%%%%%%%%%%%
\subsection{Tikz}
Packages \texttt{tikz} and \texttt{pgfplots} offer an absolutely scary plethora of features.
A select few are presented here%
\footnote{%
	By the way, footnotes look like this.%
}.
%%%%%%%%%%%%%%%%%%%%%%%%%%%%%%%%%%%%%%%%%%%%%%%%%%%%%%%%%%%%%%%%%%%%%%%%%%%%%%%%%%%%%%%%%%%
\paragraph{Drawing over Bitmaps}
When having to rely on bitmaps, once might still want to add additional info.
This can be done directly in \LaTeX{}, profiting from all the usual features.
In the example here, this is of course the retaining of the text font, but also usage of the wonderful \texttt{contour} package to draw legible black-on-white (or vice-versa) text.
An example is shown in \cref{fig:mtu_turbo}.
\begin{figure}
\ffigbox[\FBwidth]
{%
	\begin{tikzpicture}% https://tex.stackexchange.com/a/9562/120853
	[every path/.style = {draw, line width = 5pt, rounded corners, white},
	font = \sffamily,
	]
	\node[anchor=south west,inner sep=0] (img) at (0,0) {\includegraphics[width=0.7\textwidth]{mtu_turbo.jpg}};
	\begin{scope}[x={(img.south east)}, y={(img.north west)}]

	% Air Intake in the right:
	\node at (0.85, 0.55) {\ctrb{Air In}};

	% Turbine:
	\node[outer sep = 10pt] (turbine) at (0.45, 0.45) {};
	\node[below left = of turbine] (turbinetext) {\ctrb{Turbine Wheel}};
	\draw[-stealth] (turbinetext) to[out = 0, in = -120] (turbine);

	% Shaft:
	\node[outer sep = 5pt] (shaft) at (0.525, 0.43) {};
	\node[above = of shaft] (shafttext) {\ctrb{Shaft}};
	\draw[-stealth] (shafttext) to (shaft);

	% Compressor:
	\node[outer sep = 10pt] (compressor) at (0.6, 0.4) {};
	\node[below right = of compressor] (compressortext) {\ctrb{Impeller}};
	\draw[-stealth] (compressortext) to[out = 180, in = -45] (compressor);

	%\draw[help lines,xstep=.1,ystep=.1] (0,0) grid (1,1);% Uncomment block for coordinate help lines
	%	\foreach \x in {0,1,...,9} { \node [anchor=north] at (\x/10,0) {0.\x};}
	%	\foreach \y in {0,1,...,9} { \node [anchor=east] at (0,\y/10) {0.\y};}
	\end{scope}
	\end{tikzpicture}
}%
{%
	\caption[Turbocharger Rendering]%
	{
		Digital rendering of a \emph{Rolls-Royce Power Systems/MTU} turbocharger unit.
		Note the two-stage (arrangement in series) setup%
	}
	\label{fig:mtu_turbo}
	\floatfoot{\adaptedfrom{} \autocite{rolls-royce_power_systems_ag_mtu_2011}}
}
\end{figure}
%%%%%%%%%%%%%%%%%%%%%%%%%%%%%%%%%%%%%%%%%%%%%%%%%%%%%%%%%%%%%%%%%%%%%%%%%%%%%%%%%%%%%%%%%%%
\paragraph{Direct plotting}
If you rely on tools like \texttt{matlab2tikz}, maybe this is for you.
We can plot \textit{directly} into \LaTeX, without having to import outside data in the form of \texttt{*.csv}-files or automatically generated Tikz-pictures.
Still, anything more than polynomials is probably still too much.
While the functionality is limited, it may still save a lot of time and headaches.
This is demonstrated in \cref{fig:plotting_in_latex,fig:plotting_in_latex_tufte}.
\begin{figure}\ContinuedFloat*% We can continue floats across pages like this
\fcapside[\FBwidth]
{%
\caption{
	Caloric parameters of air.
	Avoid legends and put info where it belongs, improving legibility (less back-and-forth for the eye)%
	}%
\label{fig:plotting_in_latex}%
\floatfoot{See \autocite[15]{dixon_fluid_2014}}%
}%
{%
\begin{tikzpicture}
	\begin{axis}[%
	plotstyleNarrow,%
	axis y line*=right,%
	axis x line=none,%
	grid=none,
	ylabel={\gls{adexp}},% Using \ensuremath{} for each symbol, we don't have to use math mode here
	y unit = {-},
	]
		\addplot+[domain=20:400]{cpm(x)/(cpm(x)-8.3145)} node [pos=0.3, fill=white, inner sep=0.5pt, sloped] {\gls{adexp}};
	\end{axis}
	\begin{axis}[% https://tex.stackexchange.com/a/31504/120853
	plotstyleNarrow,%
	xlabel={\gls{c_temperature}},%
	x unit = {\degreeCelsius},
	ylabel={\gls{specheatcappres}},
	y unit = {\joule\per\kilogram\per\kelvin},
	cycle list shift=1,
	]
		\addplot+[domain=20:400]{cpm(x)/0.0289524} node [pos=0.3, fill=white, inner sep=0.5pt, sloped] {\gls{specheatcappres}};%
	\end{axis}
\end{tikzpicture}
}%
\end{figure}

\begin{figure}\ContinuedFloat
\fcapside[\FBwidth]
{%
	\caption{Same as \cref{fig:plotting_in_latex} in hip and \enquote{\textit{Tufte}-like}}
	\label{fig:plotting_in_latex_tufte}
}%
{%
\begin{tikzpicture}
	\begin{axis}[%
	minimalistic,%
	ylabel={\gls{specheatcappres}},
	y unit = {\joule\per\kilogram\per\kelvin},
	xlabel={\gls{temperature}},%
	x unit = {\kelvin},
	domain=300:700,
	ymin = 1000,
	ymax = 1100,
	ytick = {1000, 1050, 1100},% Specify manually due to weird rounding
	]
		\addplot+{cpm(x-273.15)/0.0289524} node [pos = 0.3,fill=white,inner sep=0.5pt, sloped] {\gls{specheatcappres}};%
	\end{axis}%
	\begin{axis}[% https://tex.stackexchange.com/a/31504/120853
	minimalistic,%
	axis y line*=right,%
	axis x line=none,%
	ylabel={\gls{adexp}},%
	y unit = {-},
	cycle list shift=1,
	domain = 300:700,
	ymin = 1.35,
	ymax = 1.4,
	]
		\addplot+{cpm(x-273.15)/(cpm(x-273.15)-8.3145)} node [pos = 0.3,fill=white,inner sep=0.5pt, sloped] {\gls{adexp}};
	\end{axis}
\end{tikzpicture}
}%
\end{figure}
%%%%%%%%%%%%%%%%%%%%%%%%%%%%%%%%%%%%%%%%%%%%%%%%%%%%%%%%%%%%%%%%%%%%%%%%%%%%%%%%%%%%%%%%%%%
\paragraph{From a file}
As discussed, often we'd want to plot data from files.
The better behaved the CSV file is (meaningful headers, no junk rows), the easier that is.
In \cref{fig:diffuser}, we only have to specify \iecfeg{e.g.} \verb|y=M| and the column corresponding to that header is automatically chosen, with no confusion about indices/numbers.
\begin{figure}
\ffigbox[\FBwidth]
{
	\caption{Some CSV data for a diffuser}
	\label{fig:diffuser}
}%
{%
\pgfplotstableread{./data/diffuser.csv}{\diffusertable}%
\begin{tikzpicture}
	\begin{axis}%
	[%
	minimalistic,%
	axis y line*=left,%
	xlabel={\(\gls{radius}/\gls{radius}_{2}\)},%
	x unit = {-},
	ylabel={\gls{mach}, \(\gls{pressure}/\gls{pressure}_{2}\), \(\gls{temperature}/\gls{temperature}_{2}\), \(\gls{density}/\gls{density}_{2}\)},%
	y unit = {-},
	table/x={R_pres},%
	ymin=0.4,
	ymax=1.3,
	ytick={0.4, 0.7, 1, 1.3},
	xmin=1,
	xmax=1.6
	]%
		% Do this manually, node macro expansion in foreach/invokeforeach is weird
		\addplot+ table [y=M] {\diffusertable} node [pos=0.2, fill=white] {\gls{mach}};
		\addplot+ table [y=Pi] {\diffusertable} node [pos=0.9, fill=white] {\(\gls{pressure}/\gls{pressure}_{2}\)};
		\addplot+ table [y=Theta] {\diffusertable} node [pos=0.7, fill=white] {\(\gls{temperature}/\gls{temperature}_{2}\)};
		\addplot+ table [y=Rho] {\diffusertable} node [pos=0.8, fill=white] {\(\gls{density}/\gls{density}_{2}\)};
	\end{axis}%
	\begin{axis}%
	[%
	minimalistic,%
	axis y line*=right,%
	axis x line=none,%
	ylabel = {abs.\ flow angle \gls{angabs}},%
	y unit = {\degree},
	cycle list shift=4,%
	ymin=13,
	ymax=15,
	xmin=1,
	xmax=1.6,
	]%
		\addplot+ table [x=R_pres, y=alpha] {\diffusertable} node [pos=0.6, fill=white] {\gls{angabs}};%
	\end{axis}%
\end{tikzpicture}
}
\end{figure}
%%%%%%%%%%%%%%%%%%%%%%%%%%%%%%%%%%%%%%%%%%%%%%%%%%%%%%%%%%%%%%%%%%%%%%%%%%%%%%%%%%%%%%%%%%%
\paragraph{Tikz and Text}
We can also draw tikz content into text content using \texttt{tikzmark}.
This, and also how to use \verb|\foreach| in tikz, is illustrated in \cref{eq:tikz_in_text}.
There, usage of chemical compounds as \verb|\chcpd| is also shown.

\noindent%
\begin{minipage}{1\linewidth}
\medmuskip = 3\medmuskip% https://tex.stackexchange.com/q/83746/120853
\thickmuskip = 3\thickmuskip
{%
\begin{equation}\label{eq:tikz_in_text}
\tikzmark{c}\gls{massfr}_{\chcpd{C}} + \tikzmark{h}\gls{massfr}_{\chcpd{H}} + \tikzmark{s}\gls{massfr}_{\chcpd{S}} + \tikzmark{o}\gls{massfr}_{\chcpd{O}} + \tikzmark{n}\gls{massfr}_{\chcpd{N}} + \tikzmark{w}\gls{massfr}_{\chcpd{H2O}} + \tikzmark{a}\gls{massfr}_{\mathrm{ash}}\coloneq 1\eqend{}
\end{equation}
\begin{tikzpicture}[remember picture,overlay]
\pgfmathsetmacro{\vshiftone}{4}
\pgfmathsetmacro{\vshifttwo}{6.5}
\pgfmathsetmacro{\vshiftthree}{9}
\pgfmathsetmacro{\hshiftone}{1}
\pgfmathsetmacro{\hshifttwo}{2}
\pgfmathsetmacro{\hshiftthree}{3}

\foreach \x/\y/\a/\b in {%
	c/Carbon/\vshiftone/-\hshiftthree,%
	h/Hydrogen/\vshifttwo/-\hshifttwo,%
	s/Sulphur/\vshiftthree/-\hshiftone,%
	o/Oxygen/\vshifttwo/0,%
	n/Nitrogen/\vshiftthree/\hshiftone,%
	w/Water/\vshifttwo/\hshifttwo,%
	a/Ash/\vshiftone/\hshiftthree%
}%
{%
	\node (\x1) [below right = 0.1em and 0.4em of pic cs:\x] {};
	\node (\x2) [on grid, below right = \a ex and \b ex of \x1, anchor=north] {\y};% On grid makes positioning snappy (uses actual middle of nodes); without, even 'right=0pt' would not be centered
	\draw[-stealth] [out=90] (\x2) to [in=270](\x1);
}%
\end{tikzpicture}
}%
\vspace{11ex}% We need to use 'overlay', but this also means we lose the bounding box. Eye-ball it here, sadly.
\end{minipage}


