\addchap{Preface}

This document is not a beginner guide.
There is a wide choice of those out there already, both free and paid for.
However, what is lacking is a collection of modern, or even at least current,
best practices.
If you scouted through package documentations and
\href{https://tex.stackexchange.com/}{StackExchange}
long enough, you would eventually get an idea of what is current, idiomatic \LaTeX{}
and what is not.
This is how I learned \LaTeX{} too, so I cannot recommend any useful beginner books.
You might want to check out \href{https://www.overleaf.com/}{Overleaf} though,
an online \LaTeX{} editor%
\footnote{%
    I do not recommend using online editors.
    You are putting your hard work onto some remote, foreign server, relying on
    their ongoing availability and forfeiting the chance of understanding \LaTeX{},
    in case you have to continue locally.
    Theses containing confidential material should also not be hosted externally.
}
with a large knowledge database.

This document is an attempt at collecting best practices
--- or at least, useful approaches ---
and pointing out the old ones they could, and often should, replace.
More than most other languages, the \LaTeX{} code in circulation world wide is
quite aged.
While that code does not necessarily get \emph{worse}, it also does not exactly
age like cheese and wine would.

\paragraph{Source Code}
This document is meant to be read side\-/by\-/side with its source code.
That is why there is almost no source code in the printed output itself.
If you are curious how a certain output is achieved, navigate to the source code
itself.
This approach was chosen since, plainly, code does not lie, while annotations and
comments might.
So while the printed output will always remain true to its actual source code,
duplicating that source code so it can be read in the printed output directly
is just another vector for errors to creep in.

\paragraph{Beginning Prerequisites}
Since this document will probably still be read by people new to \LaTeX{}, there
will be a short description on how to get started below.
\LaTeX{} requires three things:
\begin{enumerate}
    \item of course, a source text file, ending in \texttt{.tex}.
        A minimal example is:

        \begin{lstlisting}[
            style=betweenpar,
            language={[LaTeX]TeX},
        ]
            \documentclass{scrartcl}
            \begin{document}
                Hello World!
            \end{document}
        \end{lstlisting}
       Note the usage of \texttt{scrartcl} over the standard \texttt{article}.
       This is a \ctanpackage{koma-script} document class.
       If you would like to know more, more on that later.
       Otherwise, just use it everywhere and profit.
    \item a \emph{distribution}, which are the compilers, packages and other
        goodies like fonts:
        \begin{itemize}
            \item \emph{Compilers} translate high\-/level source code (see the
                first point) to a different \enquote{language}.
                In our case, the other language is \abb{portable_document_format}
                source code.
                It is not human\-/readable and mostly gibberish, but a
                \abb{portable_document_format} viewer takes care of that.
            \item \emph{Packages} are bundles of ready\-/made functionalities for
                \LaTeX{}.
                There are packages for basically everything.
                The \href{https://ctan.org/}{\abb{comprehensive_tex_archive_network}},
                a \emph{package repository}, contains basically all of them.
        \end{itemize}
       UNIX-based operating systems do well with
       \href{https://www.tug.org/texlive/}{TeXLive},
       which is available as a package for most distributions.
       It is also available for Windows.
       It has a yearly release schedule.
       So there might be \href{https://tex.stackexchange.com/a/476742/120853}{bugs}
       that do not get fixed for a whole while.
       Nevertheless, I can recommend it.

       Another viable alternative is \href{https://miktex.org/}{MiKTeX}.
       It has a rolling release model, aka updates to packages are published
       whenever they are deemed ready.
       MiKTeX's \abb{graphical_user_interface} (the \emph{MiKTeX Console}) is pretty
       polished and usable, see \cref{fig:miktex_gui}.

       \begin{figure}
            \ffigbox[\FBwidth]{%
                \caption[%
                    % Do not use regular glossaries-commands in captions (or
                    % section headers etc.): if they occur in the LoC/LoT etc.,
                    % they will be expanded there already, which is unwanted.
                    MiKTeX \glsfmtlong{abb.graphical_user_interface} on Windows%
                ]{%
                    MiKTeX \abb{graphical_user_interface} on Windows%
                }%
                \label{fig:miktex_gui}%
            }{%
                % Having issued \graphicspath globally, we do not have to specify
                % the full path here. Not even a file extension is necessary.
                \includegraphics[width=0.8\textwidth]{miktex_gui}
            }
       \end{figure}

       You should hit that juicy \emph{Check for updates} at least yearly, rather
       biannually.
       \LaTeX{} is a slow world, in which files from the previous millennium might
       very well still compile and look fine.
       However, a very large share of errors are caused by out\-/of\-/date packages.
       For example, if your \LaTeX{} distribution is ancient (anything older than,
       say, three years), and you then compile a new file that installs a new
       package, you suddenly have that package in its latest version, alongside
       all the old packages.
       That will not go well long.
    \item of course, an \emph{editor}.

       Here, you are free to do whatever you want.
       I recommend \href{https://code.visualstudio.com/}{Visual Studio Code}, using its
       \href{https://marketplace.visualstudio.com/items?itemName=James-Yu.latex-workshop}%
       {\LaTeX{} Workshop} extension, which provides syntax highlighting, shortcuts
       and many other useful things.
       VSCode is among the most state\-/of\-/the\-/art editors currently available.
       Being usable for \LaTeX{} is just a nice \enquote{side\-/effect} we can take
       advantage of.

       For a more conventional, complete \abb{integrated_development_environment},
       try \href{https://www.texstudio.org/}{TeXStudio}.
       Like VSCode, it is also
       \href{https://github.com/texstudio-org/texstudio}{open source}.
       TeXStudio will cater to \SI{99}{\percent} of your \LaTeX{} needs.

       If you like to live dangerously, you can even write your \LaTeX{} in Notepad.
        Vim is not mentioned here because its users will probably have skipped this
        section\dots{}
\end{enumerate}
