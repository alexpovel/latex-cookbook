\chapter{Code Syntax Highlighting}
\label{ch:code-listings}

To properly typeset code in \LaTeX{}, we use the \ctanpackage{minted} package.
It relies on Python, and calls in outside help for syntax highlighting using Python's
\texttt{pygments} package.
As such, it requires \texttt{--shell-escape} to compile, and of course Python.
The latter can be a pain in the buttocks to set up; Docker usage is especially useful here.
Since \texttt{pygments} has nothing to do with \LaTeX{} (whose ecosystem is a sad mess),
but is instead a regular old Python package, chances are the language of your choice is
not only available but also well\-/supported!%
\footnote{%
    For example, \ctanpackage{listings}, the inferior alternative to \ctanpackage{minted},
    still had no Python 3 support (only basic 2) in 2020, at the time of originally writing this.
    At that point, Python 3 was over 10 years old already and Python 2 was end-of-life.
}

\paragraph{Color Scheme}
Note how the color scheme is the same throughout languages.
The idea is that keywords of similar importance, status or semantics are highlighted uniformly.
For example, keywords for class and function definitions, like \mintinline{python}{class}
for Python or \mintinline{matlab}{classdef} for MATLAB.%
\footnote{%
    Notice how these keywords were created using an \emph{inline} listing, like:
    \mintinline{python}{y = [file_patterns[x] for x in ["send", "help"]]}.
}
Another example are error\-/handling and -throwing keywords, which many languages
offer.
All these different types should be identified and treated equally.
\ctanpackage{minted} is a widely used package and knows about a lot of languages.
You can check it out at
\begin{center}
    \url{https://pygments.org/demo/}.
\end{center}

If the current colors are not to your liking they can be changed easily using
\ctanpackage{minted}'s \texttt{style} option.
Refer to the comments in the source code on how to see available styles.

\paragraph{References}
Individual code lines can also be referenced.
For example, we find a \mintinline{python}{return} statement on
% NOTE: referencing single "minted" lines *CANNOT* be done with cleveref, unfortunately.
% See: https://tex.stackexchange.com/q/132420/120853 . A plain old `ref` does the trick.
line \ref{line:python_return} in \cref{lst:float_example}.
That line is also highlighted, using \ctanpackage{minted}'s \texttt{highlightlines} option.

\section{Python}
The following examples are not always complete or functioning, they are only supposed
to showcase the available syntax highlighting.
The base style looks like:
\begin{minted}{python}
    def get_nonempty_line(
        lines: Iterable[str],
        last: bool = True
    ) -> str:
        if last:
            lines = reversed(lines)
        return next(line for line in lines if line.rstrip())
\end{minted}
It is intended for (small) samples of code that flow into the surrounding text.
A second feature are code listings as regular floats, as demonstrated in
\cref{lst:float_example}.
As floats, they behave like any other figure, table \iecfeg{etc}.

\begin{listing}
    \begin{minted}[highlightlines={24}]{python}
        import json
        import logging.config
        from pathlib import Path

        from resources.helpers import path_relative_to_caller_file

        ¬\phstring{\LaTeX{} can go in here: \(\sum_{i = 1}^{n} a + \frac{\pi}{2} \)}¬

        def set_up_logging(logger_name: str) -> logging.Logger:
            """Set up a logging configuration."""
            config_filepath = path_relative_to_caller_file("logger.json")  # same directory

            try:
                with open(config_filepath) as config_file:
                    config: dict = json.load(config_file)
                logging.config.dictConfig(config)
            except FileNotFoundError:
                logging.basicConfig(
                    level=logging.INFO, format="[%(asctime)s: %(levelname)s] %(message)s"
                )
                logging.warning(f"Using fallback: no logging config found at {config_filepath}")
                logger_name = __name__

            return logging.getLogger(logger_name) ¬\label{line:python_return}¬
    \end{minted}
    \caption{%
        This is a caption.
        Listings cannot be overly long since floats do not page-break% No period here!
    }
    \label{lst:float_example}
\end{listing}

The third example showcases breaking across pages, probably best suited for an appendix:
\begin{minted}{python}
    def ansi_escaped_string( ¬\phnote{A random reference: \cref{fig:censorbox}}¬
        string: str,
        *,
        effects: Union[List[str], None] = None,
        foreground_color: Union[str, None] = None,
        background_color: Union[str, None] = None,
        bright_fg: bool = False,
        bright_bg: bool = False,
    ) -> str:
        """Provides a human-readable interface to escape strings for terminal output.

        Using ANSI escape characters, the appearance of terminal output can be changed. The
        escape chracters are numerical and impossible to remember. Also, they require
        special starting and ending sequences. This function makes accessing that easier.

        Args:
            string: The input string to be escaped and altered.
            effects: The different effects to apply, e.g. underlined.
            foreground_color: The foreground, that is text color.
            background_color: The background color (appears as a colored block).
            bright_fg: Toggle whatever color was given for the foreground to be bright.
            bright_bg: Toggle whatever color was given for the background to be bright.

        Returns:
            A string with requested ANSI escape characters inserted around the input string.
        """

        def pad_sgr_sequence(sgr_sequence: str = "") -> str:
            """Pads an SGR sequence with starting and end parts.

            To 'Select Graphic Rendition' (SGR) to set the appearance of the following
            terminal output, the CSI is called as:
            CSI n m
            So, 'm' is the character ending the sequence. 'n' is a string of parameters, see
            dict below.

            Args:
                sgr_sequence: Sequence of SGR codes to be padded.
            Returns:
                Padded SGR sequence.
            """
            control_sequence_introducer = "\x1B["  # hexadecimal '1B'
            select_graphic_rendition_end = "m"  # Ending character for SGR
            return control_sequence_introducer + sgr_sequence + select_graphic_rendition_end

        # Implement more as required, see
        # https://en.wikipedia.org/wiki/ANSI_escape_code#SGR_parameters.
        sgr_parameters = {
            "underlined": 4,
        }

        sgr_foregrounds = {  # Base hardcoded mapping, all others can be derived
            "black": 30,
            "red": 31,
            "green": 32,
            "yellow": 33,
            "blue": 34, ¬\phnum{\(30 + 4\)}¬
            "magenta": 35,
            "cyan": 36,
            "white": 37,
        }

        # These offsets convert foreground colors to background or bright color codes, see
        # https://en.wikipedia.org/wiki/ANSI_escape_code#Colors
        bright_offset = 60
        background_offset = 10

        if bright_fg:
            sgr_foregrounds = {
                color: value + bright_offset for color, value in sgr_foregrounds.items()
            }

        if bright_bg:
            background_offset += bright_offset

        sgr_backgrounds = {
            color: value + background_offset for color, value in sgr_foregrounds.items()
        }

        # Chain various parameters, e.g. 'ESC[30;47m' to get white on black, if 30 and 47
        # were requested. Note, no ending semicolon. Collect codes in a list first.
        sgr_sequence_elements: List[int] = []

        if effects is not None:
            for sgr_effect in effects:
                try:
                    sgr_sequence_elements.append(sgr_parameters[sgr_effect])
                except KeyError:
                    raise NotImplementedError(
                        f"Requested effect '{sgr_effect}' not available."
                    )
        if foreground_color is not None:
            try:
                sgr_sequence_elements.append(sgr_foregrounds[foreground_color])
            except KeyError:
                raise NotImplementedError(
                    f"Requested foreground color '{foreground_color}' not available."
                )
        if background_color is not None:
            try:
                sgr_sequence_elements.append(sgr_backgrounds[background_color])
            except KeyError:
                raise NotImplementedError(
                    f"Requested background color '{background_color}' not available."
                )

        # To .join() list, all elements need to be strings
        sgr_sequence: str = ";".join(str(sgr_code) for sgr_code in sgr_sequence_elements)

        reset_all_sgr = pad_sgr_sequence()  # Without parameters: reset all attributes
        sgr_start = pad_sgr_sequence(sgr_sequence)
        return sgr_start + string + reset_all_sgr
\end{minted}

\section{MATLAB}

This section contains example code for MATLAB, for example:
\begin{minted}{matlab}
    %{
        Universal Gas Constant for SIMULINK.
    %}
    R_m = Simulink.Parameter;
        R_m.Value = 8.3144598;
        R_m.Description = 'universal gas constant';
        R_m.DocUnits = 'J/(mol*K)';
\end{minted}
Of course, floats (see \cref{lst:matlab_class_definition}) are available as well.
So are longer sections, like the following.

\begin{listing}
    \begin{minted}{matlab}
        classdef BasicClass
            properties
                Value {mustBeNumeric}
            end
            methods
                function r = roundOff(obj)
                    r = round([obj.Value],2);
                end
                function r = multiplyBy(obj,n)
                    r = [obj.Value] * n;
                end
            end
        end
    \end{minted}
    \caption[A class definition in MATLAB]{%
        A class definition in MATLAB, from \cite{mathworksCreateSimpleClass2020}%
    }
    \label{lst:matlab_class_definition}
\end{listing}

\begin{minted}{matlab}
    fullfilepath = mfilename('fullpath');
    [filepath, filename, ~] = fileparts(fullfilepath);

    %% Begin Dialogue
    %{
    Extract Data from user-specified input. Can be either a File that is run
    (an *.m-file), variables in the Base Workspace or existing data from a
    previous run (a *.MAT-file). All variables are expected to be in Table
    format, which is the data type best suited for this work. Therefore, we
    force it. No funny business with dynamically named variables or naked
    matrices allowed.
    %}
    pass_data = questdlg({'Would you like to pass existing machine data?', ...
        'Its data would be used to compute your request.', ...
        ['Regardless, note that data is expected to be in ',...
        'Tables named ''', dflt.comp, ''' and ''', dflt.turb, '''.'], '',...
        'If you choose no, existing values are used.'}, ...
        'Machine Data Prompt', btnFF, btnWS, btnNo, btnFF);

    switch pass_data
        case btnFF
            prompt = {'File name containing the Tables:', ...
                'Compressor Table name in that file:',...
                'Turbine Table name in that file:'};
            title = 'Machine Data Input';
            dims = [1 50];
            definput = {dflt.filename.data, dflt.comp, dflt.turb};
            machine_data_file = inputdlg(prompt, title, dims, definput);
            if isempty(machine_data_file)
                fprintf('[%s] You left the dialogue and function.\n',...
                    datestr(now));
                return
            end
            run(machine_data_file{1});%spills file contents into funcWS
        case btnWS
            prompt = {'Base Workspace Compressor Table name:',...
                    'Base Workspace Turbine Table name:'};
            title = 'Machine Data Input';
            dims = [1 50];
            definput = {dflt.comp, dflt.turb};
            machine_data_ws = inputdlg(prompt, title, dims, definput);
            if isempty(machine_data_ws)
                fprintf('[%s] You left the dialogue and function.\n',...
                    datestr(now));
                return
            end
        case btnNo
            boxNo = msgbox(['Looking for stats in ''', dflt.filename.stats, ...
                '''.'], 'Using Existing Data', 'help');
            waitfor(boxNo);
            try
                stats = load(dflt.filename.stats);
                stats = stats.stats;
            catch ME
                switch ME.identifier
                    case 'MATLAB:load:couldNotReadFile'
                        warning(['File ''', dflt.filename.stats, ''' not ',...
                            'found in search path. Make sure it has been ',...
                            'generated in a previous run or created ',...
                            'manually. Resorting to hard-coded data ',...
                            'for now.']);
                            a_b = 200.89;
                            x_c = 0.0012;
                            f_g = 10.0;
                    otherwise
                        rethrow(ME);
                end
            end
        case ''
            fprintf('[%s] You left the dialogue and function.\n',datestr(now));
            return
        otherwise%Only gets here if buttons are misconfigured
            error('This option is not coded, should not get here.');
    end
\end{minted}

\subsection{MATLAB/Simulink icons}

For an older project, MATLAB/Simulink vector icons were created.
They are included here at the off\-/chance that someone else might find a use for these.
\begin{itemize}
    \item \mtlbsmlkicon{matlab_object_box}
    \item \mtlbsmlkicon{matlab_struct}
    \item \mtlbsmlkicon{matlab_table}
    \item \mtlbsmlkicon{simulink_algebraic_constraint}
    \item \mtlbsmlkicon{simulink_base_workspace}
    \item \mtlbsmlkicon{simulink_configuration}
    \item \mtlbsmlkicon{simulink_data_dictionary}
    \item \mtlbsmlkicon{simulink_library_model}
    \item \mtlbsmlkicon{simulink_library}
    \item \mtlbsmlkicon{simulink_log_data}
    \item \mtlbsmlkicon{simulink_lut}
    \item \mtlbsmlkicon{simulink_model_workspace}
    \item \mtlbsmlkicon{simulink_model}
    \item \mtlbsmlkicon{simulink_referenced_model}
    \item \mtlbsmlkicon{simulink_step_block}
\end{itemize}

\section{Modelica}

Naturally, floating and all other environments and styles are also available for
Modelica.
The syntax highlighting for a few basic code samples \autocite{wikipediacontributorsModelica2021} looks like:

\begin{minted}{modelica}
    x := 2 + y;
    x + y = 3 * z;

    model FirstOrder
        parameter Real c=1 "Time constant";
        Real x (start=10) "An unknown";
    equation
        der(x) = -c*x "A first order differential equation";
    end FirstOrder;

    type Voltage = Real(quantity="ElectricalPotential", unit="V");
    type Current = Real(quantity="ElectricalCurrent", unit="A");

    connector Pin "Electrical pin"
        Voltage      v "Potential at the pin";
        flow Current i "Current flowing into the component";
    end Pin;

    model Capacitor
        parameter Capacitance C;
        Voltage u "Voltage drop between pin_p and pin_n";
        Pin pin_p, pin_n;
    equation
        0 = pin_p.i + pin_n.i;
        u = pin_p.v - pin_n.v;
        C * der(u) = pin_p.i;
    end Capacitor;

    model SignalVoltage
        "Generic voltage source using the input signal as source voltage"
        Interfaces.PositivePin p;
        Interfaces.NegativePin n;
        Modelica.Blocks.Interfaces.RealInput v(unit="V")
            "Voltage between pin p and n (= p.v - n.v) as input signal";
        SI.Current i "Current flowing from pin p to pin n";
    equation
        v = p.v - n.v;
        0 = p.i + n.i;
        i = p.i;
    end SignalVoltage;

    model Circuit
        Capacitor C1(C=1e-4) "A Capacitor instance from the model above";
        Capacitor C2(C=1e-5) "A Capacitor instance from the model above";
            ...
    equation
        connect(C1.pin_p, C2.pin_n);
            ...
    end Circuit;
\end{minted}

\section{Lua}

Files can also be read into \LaTeX{} directly.
For example, the following is some current Lua code \emph{used for this very document}:
\inputminted{lua}{./lib/example.lua}
