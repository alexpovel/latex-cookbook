\chapter{Code Listings}

To properly typeset code in \LaTeX{}, we use the package \ctanpackage{listings}.
There is a much more powerful alternative in \ctanpackage{minted}.
But with great power comes great dependencies: \ctanpackage{minted} relies on
Python, and calls in outside help for syntax highlighting using Python's
\texttt{pygments} package.
While that is a great package, the process requires \texttt{--shell-escape} to
compile, and of course Python.
For broad usage and compatibility, that is not suitable.
Lastly, what really broke the deal (for now), is that \ctanpackage{minted} is
incompatible with \ctanpackage{floatrow}, which we use a lot \autocite{egreg_minted_2017}.
Therefore, \ctanpackage{listings} it is.

However, the latter does not come with rich support for either Python or
Modelica.
Support for Python 3 syntax and its numerous built-ins was therefore added manually.
\Citeauthor{winkler_modelica-toolslistings-modelica_2020} provide a configuration
for Modelica syntax highlighting \autocite{winkler_modelica-toolslistings-modelica_2020}.
Lastly, the built-in support of MATLAB was manually extended to include all
(as of version 2020a) over 2500 base functions, as listed by \citeauthor{mathworks_matlab_2020}
\autocite{mathworks_matlab_2020}.
It also includes all \emph{keywords} as returned by running \texttt{iskeyword} in
the MATLAB prompt.

\section{Python}

The demonstrations in this chapter are done using Python, since the modifications to
\ctanpackage{listings} were focused on that.
There are examples for the other two mentioned languages later.
One feature is a code snippet style called \texttt{betweenpar}, looking like:
\begin{lstlisting}[style=betweenpar]
    def get_nonempty_line(
        lines: Iterable[str],
        last: bool = True
    ) -> str:
        if last:
            lines = reversed(lines)
        return next(line for line in lines if line.rstrip())
\end{lstlisting}
It is intended for small samples of code that flow into the surrounding text.
A second feature are code listings as regular floats, as demonstrated in
\cref{lst:float_example}.
As floats, they behave like any other figure, table \iecfeg{etc}.

\begin{lstlisting}[
    float,
    caption={%
        This is a caption.
        Listings cannot be overly long since floats do not page-break% No period here!
    },
    label={lst:float_example},
]
    import json
    import logging.config
    from pathlib import Path

    from resources.helpers import path_relative_to_caller_file

    ¬\phstring{\LaTeX{} can go in here: \(\sum_{i = 1}^{n} a + \frac{\pi}{2} \)}¬


    def set_up_logging(logger_name: str) -> logging.Logger:
        """Set up a logging configuration."""
        config_filepath = path_relative_to_caller_file("logger.json")  # same directory

        try:
            with open(config_filepath) as config_file:
                config: dict = json.load(config_file)
            logging.config.dictConfig(config)
        except FileNotFoundError:
            logging.basicConfig(
                level=logging.INFO, format="[%(asctime)s: %(levelname)s] %(message)s"
            )
            logging.warning(f"Using fallback: no logging config found at {config_filepath}")
            logger_name = __name__

        return logging.getLogger(logger_name)
\end{lstlisting}

Lastly, you can use the base \texttt{lstlisting} environment for more elaborate,
possibly very long code listings.
These can be broken across pages and are probably best suited for the appendix.
\begin{lstlisting}
    def ansi_escaped_string( ¬\phnote{A random reference: \ref{lst:example} (use `cleveref`!)}¬
        string: str,
        *,
        effects: Union[List[str], None] = None,
        foreground_color: Union[str, None] = None,
        background_color: Union[str, None] = None,
        bright_fg: bool = False,
        bright_bg: bool = False,
    ) -> str:
        """Provides a human-readable interface to escape strings for terminal output.
    
        Using ANSI escape characters, the appearance of terminal output can be changed. The
        escape chracters are numerical and impossible to remember. Also, they require
        special starting and ending sequences. This function makes accessing that easier.
    
        Args:
            string: The input string to be escaped and altered.
            effects: The different effects to apply, e.g. underlined.
            foreground_color: The foreground, that is text color.
            background_color: The background color (appears as a colored block).
            bright_fg: Toggle whatever color was given for the foreground to be bright.
            bright_bg: Toggle whatever color was given for the background to be bright.
    
        Returns:
            A string with requested ANSI escape characters inserted around the input string.
        """
    
        def pad_sgr_sequence(sgr_sequence: str = "") -> str:
            """Pads an SGR sequence with starting and end parts.
    
            To 'Select Graphic Rendition' (SGR) to set the appearance of the following
            terminal output, the CSI is called as:
            CSI n m
            So, 'm' is the character ending the sequence. 'n' is a string of parameters, see
            dict below.
    
            Args:
                sgr_sequence: Sequence of SGR codes to be padded.
            Returns:
                Padded SGR sequence.
            """
            control_sequence_introducer = "\x1B["  # hexadecimal '1B'
            select_graphic_rendition_end = "m"  # Ending character for SGR
            return control_sequence_introducer + sgr_sequence + select_graphic_rendition_end
    
        # Implement more as required, see
        # https://en.wikipedia.org/wiki/ANSI_escape_code#SGR_parameters.
        sgr_parameters = {
            "underlined": 4,
        }
    
        sgr_foregrounds = {  # Base hardcoded mapping, all others can be derived
            "black": 30,
            "red": 31,
            "green": 32,
            "yellow": 33,
            "blue": 34, ¬\phnum{\(30 + 4\)}¬
            "magenta": 35,
            "cyan": 36,
            "white": 37,
        }
    
        # These offsets convert foreground colors to background or bright color codes, see
        # https://en.wikipedia.org/wiki/ANSI_escape_code#Colors
        bright_offset = 60
        background_offset = 10
    
        if bright_fg:
            sgr_foregrounds = {
                color: value + bright_offset for color, value in sgr_foregrounds.items()
            }
    
        if bright_bg:
            background_offset += bright_offset
    
        sgr_backgrounds = {
            color: value + background_offset for color, value in sgr_foregrounds.items()
        }
    
        # Chain various parameters, e.g. 'ESC[30;47m' to get white on black, if 30 and 47
        # were requested. Note, no ending semicolon. Collect codes in a list first.
        sgr_sequence_elements: List[int] = []
    
        if effects is not None:
            for sgr_effect in effects:
                try:
                    sgr_sequence_elements.append(sgr_parameters[sgr_effect])
                except KeyError:
                    raise NotImplementedError(
                        f"Requested effect '{sgr_effect}' not available."
                    )
        if foreground_color is not None:
            try:
                sgr_sequence_elements.append(sgr_foregrounds[foreground_color])
            except KeyError:
                raise NotImplementedError(
                    f"Requested foreground color '{foreground_color}' not available."
                )
        if background_color is not None:
            try:
                sgr_sequence_elements.append(sgr_backgrounds[background_color])
            except KeyError:
                raise NotImplementedError(
                    f"Requested background color '{background_color}' not available."
                )
    
        # To .join() list, all elements need to be strings
        sgr_sequence: str = ";".join(str(sgr_code) for sgr_code in sgr_sequence_elements)
    
        reset_all_sgr = pad_sgr_sequence()  # Without parameters: reset all attributes
        sgr_start = pad_sgr_sequence(sgr_sequence)
        return sgr_start + string + reset_all_sgr
\end{lstlisting}
