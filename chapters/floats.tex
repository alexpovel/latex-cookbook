\chapter{Float Features}
In \cref{fig:censorbox,tab:main_font_examples}, we have already seen examples for floats.
These are handled by packages \package{caption} and \package{floatrow}.
A notable feature is the capability for captions the same width as the float they are attached to.
This tends to look much prettier, \iecfeg{cf.} \cref{fig:wide_caption,fig:tighter_caption}.
%%%%%%%%%%%%%%%%%%%%%%%%%%%%%%%%%%%%%%%%%%%%%%%%%%%%%%%%%%%%%%%%%%%%%%%%%%%%%%%%%%%%%%%%%%%
\begin{figure}
\includegraphics[width=0.4\textwidth]{example_image}%
\caption{Example for a regular caption, spanning the whole width since it is so long}%
\label{fig:wide_caption}%
\end{figure}
\begin{figure}
\ffigbox[\FBwidth]% Optional argument FBwidth causes graphic to be its actual size and caption to take up the remaining space. Otherwise, horizontal space is split 50/50, which doesn't make much sense.
{%
	\caption{Example for a new caption, spanning the just the width of the float it is attached to}%
	\label{fig:tighter_caption}%
}%
{%
	\includegraphics[width=0.4\textwidth]{example_image}%
}%
\end{figure}
%%%%%%%%%%%%%%%%%%%%%%%%%%%%%%%%%%%%%%%%%%%%%%%%%%%%%%%%%%%%%%%%%%%%%%%%%%%%%%%%%%%%%%%%%%%
\section{Multiple floats}
Other possibilities are rather arbitrary arrangements of sub-figures and -captions.
For this, see \cref{fig:impeller_throat}, which contains two sub-figures \cref{fig:impeller_throat_iso,fig:impeller_throat_flat}.
Multiple sub-tables are also possible, see \cref{tab:works_on_turbocharging}.
\begin{figure}
\ffigbox[\FBwidth]
{%
	\begin{subfloatrow}[2]% Default is two anyway
		\ffigbox[\FBwidth]
		{%
			\def\svgwidth{0.4\figurewidth}
			\input{./images/impeller_isometric.pdf_tex}% input is not includegraphics, so we have to give the full path
		}%
		{%
			\caption{Isometric}%
			\label{fig:impeller_throat_iso}%
		}%
		\ffigbox[\FBwidth]
		{%
			\def\svgwidth{0.3\figurewidth}
			\input{./images/inducer_throat.pdf_tex}
		}%
		{%
			\caption{Schematic}%
			\label{fig:impeller_throat_flat}%
		}%
	\end{subfloatrow}
}%
{%
	\caption[Impeller Throat]%
	{%
		Inducer throat created by geometric constraints as an example for sub-figures%
	}%
	\label{fig:impeller_throat}%
	\floatfoot{\adaptedfrom{} \autocites{shakal_centrifugal_2015}[99]{whitfield_design_1990}[535]{hayami_flow_1985}}%
}%
\end{figure}
\begin{table}
	\tiny
	\floatbox{table}%
	{%https://tex.stackexchange.com/q/295567/120853
		\begin{subfloatrow}
			\ttabbox[\FBwidth]%
			{%
				\begin{tabular}{@{}m{0.25\textwidth}m{0.15\textwidth}@{}}
					\toprule
					Author \& [Work] & Comment\\
					\midrule
					\textcite{bozza_map-based_2011} & ---\\
					\textcite{bozza_numerical_2013} & ---\\
					\textcite{bozza_theoretical_1990} & \textbf{Nozzles}\\
					\textcite{burke_modelling_2014} & ---\\
					\textcite{de_bellis_development_2018} & ---\\
					\textcite{berndt_einfluss_2009} & ---\\
					\textcite{chesse_performance_2000} & ---\\
					\textcite{cornolti_1d_2013} & ---\\
					\textcite{eriksson_modeling_2007} & ---\\
					\textcite{grigoriadis_experimentelle_2008} & ---\\
					\textcite{kech_model-based_2002} & ---\\
					\textcite{lee_simulation-based_2009} & ---\\
					\textcite{leufven_surge_2011} & ---\\
					\textcite{shaaban_part-load_2006} & ---\\
					\textcite{wahlstrom_modelling_2011} & ---\\
					\textcite{watel_matching_2010} & \mtlbsmlnk{}\\
					\textcite{yang_mixed_2010} & ---\\
					\bottomrule
				\end{tabular}
			}%
			{%
				\caption{Using Maps}%
				\label{tab:works_using_maps}%
			}%
			\ttabbox[\FBwidth]%
			{%
				\begin{tabular}{@{}m{0.25\textwidth}m{0.15\textwidth}@{}}
					\toprule
					Author \& [Work] & Comment\\
					\midrule
					\textcite{beckey_compressor_2011} & using algorithms\\
					\textcite{bergqvist_prediction_2014} & from CFD\\
					\textcite{bohn_modellierung_2002} & ---\\
					\textcite{bolz_critical_2014} & ---\\
					\textcite{elkamel_experimental_2011} & ---\\
					\textcite{ewert_modellierung_2013} & ---\\
					\textcite{freeman_compressor_2011} & \mtlb{} script\\
					\textcite{hansch_untersuchung_2010} & ---\\
					\textcite{harley_inlet_2014} & ---\\
					\textcite{luddecke_engine_2014} & ---\\
					\textcite{kurzke_correlations_2011} & ---\\
					\textcite{schwarz_considerations_2014} & ---\\
					\bottomrule
				\end{tabular}
			}%
			{%
				\caption{Computing Maps}%
				\label{tab:works_computing_maps}%
			}%
		\end{subfloatrow}
		
		\begin{subfloatrow}
			\ttabbox[\FBwidth]%
			{%
				\begin{tabular}{@{}m{0.25\textwidth}m{0.15\textwidth}@{}}
					\toprule
					Author \& [Work] & Comment\\
					\midrule
					\textcite{aungier_mean_1995} & ---\\
					\textcite{casey_method_2013} & Calibrating with Maps\\
					\textcite{chen_one-dimensional_2014} & Turbine\\
					\textcite{erickson_centrifugal_2008} & \fortran{}\\
					\textcite{ewert_modellierung_2013} & ---\\
					\textcite{gong_total_2014} & ---\\
					\textcite{gutierrez_velasquez_one_2010} & \fortran{}, 1D and 3D\\
					\textcite{japikse_turbomachinery_2009} & Overview\\
					\textcite{kamaleshaiah_improved_1988} & ---\\
					\textcite{kessel_modellbildung_2002} & 1- \& 3D\\
					\textcite{lee_dual-stage_2008} & dual-stage; \textbf{\smlnk{}, maps}\\
					\textcite{nakhjiri_vatl_2018} & ---\\
					\textcite{okhuahesogie_1-d_2014} & Differential Evolution\\
					\textcite{pelton_one-dimensional_2007} & ---\\
					\textcite{pixberg_modellbildung_2013} & \textbf{Reservoirs; Nozzles} (p.\ 42)\\
					\textcite{sakellaridis_development_2015} & ---\\
					\textcite{schiffmann_design_2010} & ---\\
					\textcite{schur_transient_2013} & \smlnk{}\\
					\textcite{serrano_model_2008} & \textbf{Reservoirs; Nozzles}\\
					\textcite{shaaban_experimental_2004} & Diabatic; Extrapolation\\
					\textcite{stuart_analysis_2014} & cites \autocite{harley_evaluation_2013}\\
					\textcite{taburri_model-based_2012} & \textbf{Reservoirs; Nozzles}\\
					\textcite{uchida_transient_2006} & ---\\
					\textcite{whitfield_preliminary_1990} & Preliminary Design\\
					\textcite{xu_empirical_2012} & ---\\
					\textcite{zahn_arbeitsspielaufgeloste_2012} & ---\\
					\textcite{zhuge_development_2009} & ---\\
					\bottomrule
				\end{tabular}
			}%
			{%
				\caption{Physical (0- \& 1D)}%
				\label{tab:works_physical}%
			}%
			\ttabbox[\FBwidth]%
			{%
				\begin{tabular}{@{}m{0.25\textwidth}m{0.15\textwidth}@{}}
					\toprule
					Author \& [Work] & Comment\\
					\midrule
					\textcite{aungier_mean_1995} & ---\\
					\textcite{doustmohammadi_experimental_2013} & ---\\
					\textcite{el-maksoud_prediction_2012} & ---\\
					\textcites{galvas_analytical_1972}{galvas_fortran_1973} & ---\\
					\textcite{gulich_disk_2003} & Disk Friction\\
					\textcite{gutierrez_velasquez_determination_2017} & ---\\
					\textcite{harley_evaluation_2013} & Recommends \autocite{oh_optimum_1997} \\
					\textcite{mohtar_increasing_2010} & ---\\
					\textcite{nakhjiri_physical_2011} & ---\\
					\textcite{oh_optimum_1997} & ---\\
					\textcite{tacconi_investigation_2018} & Cites \autocite{harley_evaluation_2013}; recommends \autocite{galvas_fortran_1973}\\
					\textcite{schneider_analytical_2015} & 2D\\
					\textcite{solaesa_analytical_2016} & ---\\
					\bottomrule
				\end{tabular}
			}%
			{%
				\caption{Loss Modelling}%
				\label{tab:works_losses}%
			}%
		\end{subfloatrow}%
	}%
	{%
		\caption[Works on zero- and one-dimensional modelling]
		{%
			Works on zero- and one-dimensional turbocharger and engine modelling%
		}%
		\label{tab:works_on_turbocharging}%
	}%
\end{table}
%%%%%%%%%%%%%%%%%%%%%%%%%%%%%%%%%%%%%%%%%%%%%%%%%%%%%%%%%%%%%%%%%%%%%%%%%%%%%%%%%%%%%%%%%%%
\section{Side-Captions}
Lastly, on occasion figures and their captions might look disproportionate in combination.
In these cases, placing a side-caption might relieve the situation, as shown in \cref{fig:sidecap}.
\begin{figure}
\fcapside[\FBwidth]%
{%
	\caption{A side caption, which may also span multiple lines like demonstrated in this rather long caption right here}
	\label{fig:sidecap}%
}%
{%
	\includegraphics[width=0.4\textwidth]{example_image}%
}%
\end{figure}
%%%%%%%%%%%%%%%%%%%%%%%%%%%%%%%%%%%%%%%%%%%%%%%%%%%%%%%%%%%%%%%%%%%%%%%%%%%%%%%%%%%%%%%%%%%
\section{Caption Positioning}
Note that the caption of \cref{tab:main_font_examples} occurs \textit{above} the table no matter the \package{caption} command's position.
Per convention, figure captions should appear below, table captions above their bodies.
This is also handled by \package{floatrow}.
Also note that there is neither a fullstop nor \textit{any} character (no space, no empty line) behind the last caption line in the source code, since dots are managed globally by the \package{caption} package.
\paragraph{Float Footer}
We also have a \verb|\floatfoot| command for all floats.
This is used to place additional info underneath the caption, primarily used for references, \iecfeg{cf.}\ \cref{fig:impeller_throat}.
%%%%%%%%%%%%%%%%%%%%%%%%%%%%%%%%%%%%%%%%%%%%%%%%%%%%%%%%%%%%%%%%%%%%%%%%%%%%%%%%%%%%%%%%%%%
%%%%%%%%%%%%%%%%%%%%%%%%%%%%%%%%%%%%%%%%%%%%%%%%%%%%%%%%%%%%%%%%%%%%%%%%%%%%%%%%%%%%%%%%%%%
\section{Tikz and pgfplots}
Packages \package{tikz} and \package{pgfplots} offer an absolutely scary plethora of features.
A select few are presented here.
%%%%%%%%%%%%%%%%%%%%%%%%%%%%%%%%%%%%%%%%%%%%%%%%%%%%%%%%%%%%%%%%%%%%%%%%%%%%%%%%%%%%%%%%%%%
\subsection{Drawing over Bitmaps}
When having to rely on bitmaps, once might still want to add additional info.
This can be done directly in \LaTeX{}, profiting from all the usual features.
In the example here, this is of course the retaining of the text font, but also usage of the wonderful \package{contour} package to draw legible black-on-white (or vice-versa) text.
An example is shown in \cref{fig:mtu_turbo}.
\begin{figure}
\ffigbox[\FBwidth]
{%
	\begin{tikzpicture}% https://tex.stackexchange.com/a/9562/120853
	[every path/.style = {draw, line width = 5pt, rounded corners, white},
	font = \sffamily,
	]
	\node[anchor=south west,inner sep=0] (img) at (0,0) {\includegraphics[width=0.7\textwidth]{mtu_turbo.jpg}};
	\begin{scope}[x={(img.south east)}, y={(img.north west)}]

	% Air Intake in the right:
	\node at (0.85, 0.55) {\ctrb{Air In}};

	% Turbine:
	\node[outer sep = 10pt] (turbine) at (0.45, 0.45) {};
	\node[below left = of turbine] (turbinetext) {\ctrb{Turbine Wheel}};
	\draw[-stealth] (turbinetext) to[out = 0, in = -120] (turbine);

	% Shaft:
	\node[outer sep = 5pt] (shaft) at (0.525, 0.43) {};
	\node[above = of shaft] (shafttext) {\ctrb{Shaft}};
	\draw[-stealth] (shafttext) to (shaft);

	% Compressor:
	\node[outer sep = 10pt] (compressor) at (0.6, 0.4) {};
	\node[below right = of compressor] (compressortext) {\ctrb{Impeller}};
	\draw[-stealth] (compressortext) to[out = 180, in = -45] (compressor);

%	\draw[help lines,xstep=.1,ystep=.1] (0,0) grid (1,1);% Uncomment block for coordinate help lines
%		\foreach \x in {0,1,...,9} { \node [anchor=north] at (\x/10,0) {0.\x};}
%		\foreach \y in {0,1,...,9} { \node [anchor=east] at (0,\y/10) {0.\y};}
	\end{scope}
	\end{tikzpicture}
}%
{%
	\caption[Turbocharger Rendering]%
	{
		Digital renders have to be implemented as bitmaps.
		You can actually vectorize images like this one, too, with decent success.
		However, the resulting file will contain an absurd number of paths and can be multiple megabytes in size.
		We can draw onto bitmaps using Tikz.
	}
	\label{fig:mtu_turbo}
	\floatfoot{\adaptedfrom{} \autocite{rolls-royce_power_systems_ag_mtu_2011}.
				In the current \texttt{*.bib} file, this reference was automatically generated by Zotero as \textbf{\texttt{@artwork}}.
				While that is okay, \package{biblatex} will throw a warning that there is no driver for such a type.
				It falls back to \textbf{\texttt{@misc}}, which works just fine%
	}
}
\end{figure}
%%%%%%%%%%%%%%%%%%%%%%%%%%%%%%%%%%%%%%%%%%%%%%%%%%%%%%%%%%%%%%%%%%%%%%%%%%%%%%%%%%%%%%%%%%%
\subsection{Direct plotting}
If you rely on tools like \texttt{matlab2tikz}, maybe this is for you.
We can plot \textit{directly} into \LaTeX, without having to import outside data in the form of \texttt{*.csv}-files or automatically generated Tikz-pictures.
Still, anything more than polynomials is probably still too much.
While the functionality is limited, it may still save a lot of time and headaches.
This is demonstrated in \cref{fig:plotting_in_latex,fig:plotting_in_latex_tufte}.
\begin{figure}\ContinuedFloat*% We can continue floats across pages like this
\fcapside[\FBwidth]
{%
\caption{
	Caloric parameters of air.
	\textbf{Avoid legends} and put info where it belongs, improving legibility (less back-and-forth for the eye)%
	}%
\label{fig:plotting_in_latex}%
\floatfoot{See \autocite[15]{dixon_fluid_2014}}%
}%
{%
\begin{tikzpicture}
	\begin{axis}[%
	plotstyleNarrow,%
	axis y line*=right,%
	axis x line=none,%
	grid=none,
	ylabel={\gls{adexp}},% Using \ensuremath{} for each symbol, we don't have to use math mode here
	y unit = {-},
	]
		\addplot+[domain=20:400]{cpm(x)/(cpm(x)-8.3145)} node [pos=0.3, fill=white, inner sep=0.5pt, sloped] {\gls{adexp}};
	\end{axis}
	\begin{axis}[% https://tex.stackexchange.com/a/31504/120853
	plotstyleNarrow,%
	xlabel={\gls{c_temperature}},%
	x unit = {\degreeCelsius},
	ylabel={\gls{specheatcappres}},
	y unit = {\joule\per\kilogram\per\kelvin},
	cycle list shift=1,
	]
		\addplot+[domain=20:400]{cpm(x)/0.0289524} node [pos=0.3, fill=white, inner sep=0.5pt, sloped] {\gls{specheatcappres}};%
	\end{axis}
\end{tikzpicture}
}%
\end{figure}

\begin{figure}\ContinuedFloat
\fcapside[\FBwidth]
{%
	\caption[Tufte-like plot]{
		Same as \cref{fig:plotting_in_latex} in hip and \enquote{\textit{Tufte}-like}.
		I think it looks gorgeous.
		For more info, refer to its namesake, \name{Tufte}{Edward}.
		Note how this is a \emph{continued} float%
	}
	\label{fig:plotting_in_latex_tufte}
}%
{%
\begin{tikzpicture}
	\begin{axis}[%
	minimalistic,%
	ylabel={\gls{specheatcappres}},
	y unit = {\joule\per\kilogram\per\kelvin},
	xlabel={\gls{abs_temperature}},%
	x unit = {\kelvin},
	domain=300:700,
	ymin = 1000,
	ymax = 1100,
	ytick = {1000, 1050, 1100},% Specify manually due to weird rounding
	]
		\addplot+{cpm(x-273.15)/0.0289524} node [pos = 0.3,fill=white,inner sep=0.5pt, sloped] {\gls{specheatcappres}};%
	\end{axis}%
	\begin{axis}[% https://tex.stackexchange.com/a/31504/120853
	minimalistic,%
	axis y line*=right,%
	axis x line=none,%
	ylabel={\gls{adexp}},%
	y unit = {-},
	cycle list shift=1,
	domain = 300:700,
	ymin = 1.35,
	ymax = 1.4,
	]
		\addplot+{cpm(x-273.15)/(cpm(x-273.15)-8.3145)} node [pos = 0.3,fill=white,inner sep=0.5pt, sloped] {\gls{adexp}};
	\end{axis}
\end{tikzpicture}
}%
\end{figure}
%%%%%%%%%%%%%%%%%%%%%%%%%%%%%%%%%%%%%%%%%%%%%%%%%%%%%%%%%%%%%%%%%%%%%%%%%%%%%%%%%%%%%%%%%%%
\subsection{Plotting from files}
As discussed, often we'd want to plot data from files.
The better behaved the CSV file is (meaningful headers, no junk rows), the easier that is.
In \cref{fig:diffuser}, we only have to specify \iecfeg{e.g.} \verb|y=M| and the column corresponding to that header is automatically chosen, with no confusion about indices/numbers.
\begin{figure}
\ffigbox[\FBwidth]
{
	\caption{A plot from CSV data for a diffuser}
	\label{fig:diffuser}
}%
{%
\pgfplotstableread{./data/diffuser.csv}{\diffusertable}%
\begin{tikzpicture}
	\begin{axis}%
	[%
	minimalistic,%
	axis y line*=left,%
	xlabel={\(\gls{radius}/\gls{radius}_{2}\)},%
	x unit = {-},
	ylabel={\gls{mach}, \(\gls{pressure}/\gls{pressure}_{2}\), \(\gls{abs_temperature}/\gls{abs_temperature}_{2}\), \(\gls{density}/\gls{density}_{2}\)},%
	y unit = {-},
	table/x={R_pres},%
	ymin=0.4,
	ymax=1.3,
	ytick={0.4, 0.7, 1, 1.3},
	xmin=1,
	xmax=1.6
	]%
		% Do this manually, node macro expansion in foreach/invokeforeach is weird
		\addplot+ table [y=M] {\diffusertable} node [pos=0.2, fill=white] {\gls{mach}};
		\addplot+ table [y=Pi] {\diffusertable} node [pos=0.9, fill=white] {\(\gls{pressure}/\gls{pressure}_{2}\)};
		\addplot+ table [y=Theta] {\diffusertable} node [pos=0.7, fill=white] {\(\gls{abs_temperature}/\gls{abs_temperature}_{2}\)};
		\addplot+ table [y=Rho] {\diffusertable} node [pos=0.8, fill=white] {\(\gls{density}/\gls{density}_{2}\)};
	\end{axis}%
	\begin{axis}%
	[%
	minimalistic,%
	axis y line*=right,%
	axis x line=none,%
	ylabel = {abs.\ flow angle \gls{angabs}},%
	y unit = {\degree},
	cycle list shift=4,%
	ymin=13,
	ymax=15,
	xmin=1,
	xmax=1.6,
	]%
		\addplot+ table [x=R_pres, y=alpha] {\diffusertable} node [pos=0.6, fill=white] {\gls{angabs}};%
	\end{axis}%
\end{tikzpicture}
}
\end{figure}
%%%%%%%%%%%%%%%%%%%%%%%%%%%%%%%%%%%%%%%%%%%%%%%%%%%%%%%%%%%%%%%%%%%%%%%%%%%%%%%%%%%%%%%%%%%
\subsection{Tikz and Text}
We can also draw tikz content into text content using \verb|\tikzmark|.
This, and also how to use \verb|\foreach| in tikz, is illustrated in \cref{eq:tikz_in_text}.
There, usage of chemical compounds as \verb|\chcpd| is also shown.

\noindent%
\begin{minipage}{1\linewidth}
\medmuskip = 3\medmuskip% https://tex.stackexchange.com/q/83746/120853
\thickmuskip = 3\thickmuskip
{%
\begin{equation}\label{eq:tikz_in_text}
\tikzmark{c}\gls{massfr}_{\chcpd{C}} + \tikzmark{h}\gls{massfr}_{\chcpd{H}} + \tikzmark{s}\gls{massfr}_{\chcpd{S}} + \tikzmark{o}\gls{massfr}_{\chcpd{O}} + \tikzmark{n}\gls{massfr}_{\chcpd{N}} + \tikzmark{w}\gls{massfr}_{\chcpd{H2O}} + \tikzmark{a}\gls{massfr}_{\mathrm{ash}}\coloneq 1\eqend{}
\end{equation}
\begin{tikzpicture}[remember picture,overlay]
\pgfmathsetmacro{\vshiftone}{4}
\pgfmathsetmacro{\vshifttwo}{6.5}
\pgfmathsetmacro{\vshiftthree}{9}
\pgfmathsetmacro{\hshiftone}{1}
\pgfmathsetmacro{\hshifttwo}{2}
\pgfmathsetmacro{\hshiftthree}{3}

\foreach \x/\y/\a/\b in {%
	c/Carbon/\vshiftone/-\hshiftthree,%
	h/Hydrogen/\vshifttwo/-\hshifttwo,%
	s/Sulphur/\vshiftthree/-\hshiftone,%
	o/Oxygen/\vshifttwo/0,%
	n/Nitrogen/\vshiftthree/\hshiftone,%
	w/Water/\vshifttwo/\hshifttwo,%
	a/Ash/\vshiftone/\hshiftthree%
}%
{%
	\node (\x1) [below right = 0.1em and 0.4em of pic cs:\x] {};
	\node (\x2) [on grid, below right = \a ex and \b ex of \x1, anchor=north] {\y};% On grid makes positioning snappy (uses actual middle of nodes); without, even 'right=0pt' would not be centered
	\draw[-stealth] [out=90] (\x2) to [in=270](\x1);
}%
\end{tikzpicture}
}%
\vspace{11ex}% We need to use 'overlay', but this also means we lose the bounding box. Eye-ball it here, sadly.
\end{minipage}

%%%%%%%%%%%%%%%%%%%%%%%%%%%%%%%%%%%%%%%%%%%%%%%%%%%%%%%%%%%%%%%%%%%%%%%%%%%%%%%%%%%%%%%%%%%
\subsection{Regular Tikz pictures}
Tikz really is \textit{not} meant for drawing.
The more free-form images shown here were created in InkScape.
Still, \enquote{drawing} in Tikz is much preferred and nicer when the images are somewhat programmatic, aka there's a lot of \SI{90}{\degree} corners, equal distances, and everything is a bit \enquote{block-like}, repetitive.
Then, your chances of success using Tikz are much improved, for it is suited well then.
For example, a small file structure diagram:

\begin{center}
\begin{tikzpicture}[%http://www.texample.net/tikz/examples/filesystem-tree/
	grow via three points={one child at (0.5,-0.7) and
		two children at (0.5,-0.7) and (0.5,-1.4)},
	edge from parent path={(\tikzparentnode.south) |- (\tikzchildnode.west)},
	every node/.style={draw=black,thick,anchor=west,fill=g5},
	font=\ttfamily,%
	]%
	\node {\faFileO{} parent.py}
	child { node {\faFileO{} constants.py}}
	child { node {\faFileO{} parameters.py}}
	child { node {\faFileO{} handling.py}}
	child { node {\faFileO{} performance\_maps.py}};
\end{tikzpicture}
\end{center}

Note how \verb|tikzpicture| environments don't have to be contained in floats.
A second example is shown in \cref{fig:tikz_diagram}.

\begin{figure}
\ffigbox[0.95\linewidth]
{%
	\begin{tikzpicture}%https://tex.stackexchange.com/a/166254/120853
	[%
	every path/.style={thick},%https://tex.stackexchange.com/a/302931/120853
	]%
	
	% Row of Model Blocks
	\node[simblock, minimum height = 5ex] (outlet) {Outlet};
	\node[simblock, minimum height = 5ex, right = of outlet] (turbine) {Turbine};
	\node[simblock, minimum height = 5ex, right = of turbine] (compressor) {Compressor};
	\node[simblock, minimum height = 5ex, right = of compressor] (inlet) {Inlet};
	\draw[->] (outlet) -- (turbine) node [midway, above, font = \footnotesize] {feeds};
	\draw[->] (turbine) -- (compressor) node [midway, above, font = \footnotesize, align = center] (shaft) {via\\shaft};
	\draw[->] (compressor) -- (inlet) node [midway, above, font = \footnotesize] {feeds};
	
	% Division Block underneath Engine
	\node[below = 12ex of inlet.south, draw= none] (div) {};
	\node[below = 4ex of div, draw = none] (mult) {};
	\node[simblock, fit = (mult) (div), minimum height = 8ex] (multdiv) {};
	\node at (div) {\(\div\)};
	\node at (mult) {\(\ast\)};
	
	% Connect Inlet and Division part
	\draw[->] (inlet.east) -- ++ (+1, 0)  node [pos=1, above] {\(p_{\gls{inlet}}(\tau)\)} |- ($(multdiv.east)!0.5!(multdiv.north east)$);
	
	% p min block next to multiplication
	\draw[<-] ($(multdiv.east)!0.5!(multdiv.south east)$) -- ++ (1, 0) node[simblock] {\(p_{\gls{air},\gls{ratedeng}}\)};
	
	\node [circle, fill, inner sep = 0.3ex, left = 0.7em of multdiv] (split1) {};
	\draw (multdiv) -- (split1);
	
	% Bias and inverted Bias blocks
	\node[simblock, left = of div] (biasinv) {\(1 - u\)};
	\node[simblock, left = of mult] (bias) {\(u - 1\)};
	\draw[->] (split1) |- (bias);
	\draw[->] (split1) |- (biasinv);
	
	% Split again
	\node [circle, fill, inner sep = 0.3ex, left = 1em of biasinv] (split2) {};
	\draw (biasinv) -- (split2);
	
	\node[simblock, left = 1em of split2, triangle, shape border rotate = 90] (gainint) {\small\(I\)};
	\draw[->] (split2) -- (gainint);
	
	\node[simblock, above = of gainint, triangle, shape border rotate = 90] (gainp) {\small\(P\)};
	\draw[->] (split2) |- (gainp);
	
	% Intersection between Integer Gain and Multi/Div block as helper
	\coordinate (help1) at (gainint|-multdiv);
	
	% Integer Text, but don't draw
	\node[draw = none, left = 4em of help1, minimum height = 8ex] (int) {\(\frac{KTs}{z - 1}\)};
	
	% Falling Edge Sign with Arrow
	\draw ($(int.east)!0.5!(int.south east)$) -- ++ (0.5em, 0) -- ++ (0, -0.7em) -- ++ (0.5em, 0) coordinate (intsign_end);
	\draw[thin, ->] ($(int.east)!0.5!(int.south east)$) -- ++ (0.5em, 0) -- ++ (0, -0.5em);
	
	% Move fitting to background so it does not overwrite the nodes it fits to
	\begin{pgfonlayer}{background}
	\node[simblock, fit = (int.north west) (intsign_end)] (int_block) {};
	\end{pgfonlayer}
	
	\draw[->] (bias) -- (int_block.east|-mult) node [midway, below, align = center, font = \footnotesize] {falling edge\\reset};
	\draw[->] (gainint) -- (int_block.east|-div);
	
	\node [simblock, circle, left = of int_block] (sum) {};
	\draw[->] (gainp) -| (sum) node [pos = 0.9, right] {\(+\)};
	\draw[->] (int_block) -- (sum) node [pos = 0.8, below] {\(+\)};
	
	\node[simblock, left = of sum] (biasinvout) {\(1- u\)};
	\draw[->] (sum) -- (biasinvout);
	
	\node[simblock, left = of outlet] (times) {\(\times\)};
	\draw[->] (times) -- (outlet) node [midway, above, font = \footnotesize] {\(\dot{m}_{\gls{turbine}}\)};
	
	\node[simblock, below = of times] (sat) {\(\num{0.1} \leq u \leq \num{1}\)};
	\draw[->] (biasinvout) -| (sat);
	\draw[->] (sat) -- (times) node [midway, left] {\(\mathbf{\gls{wastegate_position}(\tau)}\)};
	
	\node[simblock, above = of outlet, minimum height = 5ex] (eng) {Engine};
	
	\draw[->] (inlet) |- (eng) node [pos = 0.8, above, font = \footnotesize] {feeds};
	\draw[->] (eng) -| (times) node [pos = 0.7, left, font = \footnotesize] {\(\dot{m}_{\gls{exhaust}}\)};
	\end{tikzpicture}
}%
{%
	\caption[A Tikz diagram]{Wastegate implementation in a feedback-loop in \smlnk{} as an example for a Tikz diagram}%
	\label{fig:tikz_diagram}%
}%
\end{figure}
%%%%%%%%%%%%%%%%%%%%%%%%%%%%%%%%%%%%%%%%%%%%%%%%%%%%%%%%%%%%%%%%%%%%%%%%%%%%%%%%%%%%%%%%%%%
\subsection{InkScape}
Having seen what Tikz is good for, \cref{fig:inkscape_example1,fig:inkscape_example2} are good examples for when InkScape might be the better choice: 3D drawings with curves.
\Cref{fig:impeller_throat_iso} is a bitmap turned vector; InkScape can detect edges/contrasts in bitmaps and replicate those lines in a vector graphic.
While InkScape is really awkward to use at times, and severely lacking behind its commercial competitors (which is fine), it has 3D perspective tools that can also handle these cases.
That being said, Tikz is always the preferred method if so feasible.

InkScape will let you create freely scalable vector graphics (\texttt{*.svg}).
They may then be exported as \texttt{*.pdf}; now, that is only half the story.
The vector graphic's text part is exported to a plain text (\iecfeg{i.e.}\ \TeX) file, as \verb|*.pdf_tex|.
On the latter, we then call \verb|\input| within \LaTeX{}, and that file in turn embeds the previously created \texttt{*.pdf} file.

\paragraph{Single-Page PDFs}
If you aim to reuse just the \texttt{*.pdf}, without the text, outside of \LaTeX{}, perhaps on your website, then you will want it to really only contain \emph{one} page.
However, InkScape may create multi-page files.
Selecting all text elements in the \texttt{*.svg} file and moving them to the top-most layer (press \texttt{POS1}) will ensure that the generated \texttt{*.pdf} will have just one page containing the entire image.
\begin{figure}\ContinuedFloat*
\ffigbox[\FBwidth]%
{%
	\caption{InkScape is likely a better choice for images like this one}%
	\label{fig:inkscape_example1}%
	\floatfoot{\adaptedfrom{} \autocite[47]{stanitz_one-dimensional_1952}}%
}%
{%
	\small
	\def\svgwidth{0.6\figurewidth}
	\input{./images/diffuser_force_balance.pdf_tex}
}%
\end{figure}%
\begin{figure}\ContinuedFloat
	\fcapside[\FBwidth]%
	{%
		\caption{Another example for InkScape usage}%
		\label{fig:inkscape_example2}%
		\floatfoot{\adaptedfrom{} \autocite[30]{guzzella_introduction_2010}}%
	}%
	{%
		\small
		\def\svgwidth{0.6\figurewidth}
		\input{./images/gas_volume.pdf_tex}
	}%
\end{figure}%
%%%%%%%%%%%%%%%%%%%%%%%%%%%%%%%%%%%%%%%%%%%%%%%%%%%%%%%%%%%%%%%%%%%%%%%%%%%%%%%%%%%%%%%%%%%
\section{Illustrations}
As a special gimmick, there is an environment for illustrations.
It may be useless now, but you can alter it to suit your needs; the skeleton is there for you.
It mainly behaves like the other floats: it\dots floats (if you allow it to) and also has its own list.
For an example, see \cref{ill:example}.

\begin{illustration}{I am a useless box}\label{ill:example}
	There can be pretty much any content in here.
	Math works, as we can see
	\begin{equation}
	1 = 1!
	\end{equation}
	still holds true after all these years.
	Inserting other floats in here will cause \LaTeX{} to have a massive fit.
	We circumvent this by setting the \verb|[H]| flag, saying \enquote{Oh hell yeah, we do want this float right here}.
	\textbf{In any other context, setting any such flag is considered poor style.}
	Well, by me at least.
	You \textbf{will} be setting these flags prematurely and then run into countless issues of placement and \textbf{very poor} spacing.
	For the love of God, let \LaTeX{} do its job in placing the floats.
	Truth be told, they will on occasion not be placed where you need or want them.
	Keep working.
	At the very end, when all is done, go ahead and change the few misplaced floats manually, by shoving them about in the source code (still not using \verb|htb!| flags).
	Then, you're done, with minimal pain and maximum usage of \LaTeX{} spacing/placing capabilities.
	Here is such a float within a float:
	\begin{figure}[H]
		\fcapside[\FBwidth]
		{%
			\caption{Side-captions are still possible. So are labels}
			\label{fig:inside_float}
		}%
		{%
			\includegraphics[width=0.6\linewidth]{example_image}
		}
	\end{figure}
	This box even breaks across pages if so required.
	Should this turn out ugly, some manual action is certainly required.
\end{illustration}