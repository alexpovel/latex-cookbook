% This file includes various thermodynamic shapes, usable in regular TikZ graphics.

\pgfdeclareshape{valve}{%
    \inheritsavedanchors[from=rectangle]% Inherit from rectangle, should be good enough
    \inheritanchorborder[from=rectangle]
    \inheritanchor[from=rectangle]{north}
    \inheritanchor[from=rectangle]{north west}
    \inheritanchor[from=rectangle]{north east}
    \inheritanchor[from=rectangle]{center}
    \inheritanchor[from=rectangle]{west}
    \inheritanchor[from=rectangle]{east}
    \inheritanchor[from=rectangle]{mid}
    \inheritanchor[from=rectangle]{mid west}
    \inheritanchor[from=rectangle]{mid east}
    \inheritanchor[from=rectangle]{base}
    \inheritanchor[from=rectangle]{base west}
    \inheritanchor[from=rectangle]{base east}
    \inheritanchor[from=rectangle]{south}
    \inheritanchor[from=rectangle]{south west}
    \inheritanchor[from=rectangle]{south east}

    \backgroundpath{%
        % Store lower left in xa/ya and upper right in xb/yb
        \southwest \pgf@xa=\pgf@x \pgf@ya=\pgf@y
        \northeast \pgf@xb=\pgf@x \pgf@yb=\pgf@y

        % Construct main path, basically 'two triangles touching'
        \pgfpathmoveto{\pgfpoint{\pgf@xa}{\pgf@ya}}
            \pgfpathlineto{\pgfpoint{\pgf@xb}{\pgf@yb}}
            \pgfpathlineto{\pgfpoint{\pgf@xa}{\pgf@yb}}
            \pgfpathlineto{\pgfpoint{\pgf@xb}{\pgf@ya}}
        \pgfpathclose
    }
}

\pgfdeclareshape{threeway valve}{
    % This is asymetrical and stuff, so inheriting from the base valve is a bit hopeless, I don't know how to do it properly

    \inheritsavedanchors[from=rectangle]% Inherit from rectangle, should be good enough
    \inheritanchorborder[from=rectangle]

    \inheritanchor[from=rectangle]{north west}
    \inheritanchor[from=rectangle]{north east}
    \inheritanchor[from=rectangle]{center}
    \inheritanchor[from=rectangle]{west}
    \inheritanchor[from=rectangle]{east}
    \inheritanchor[from=rectangle]{mid}
    \inheritanchor[from=rectangle]{mid west}
    \inheritanchor[from=rectangle]{mid east}
    \inheritanchor[from=rectangle]{base}
    \inheritanchor[from=rectangle]{base west}
    \inheritanchor[from=rectangle]{base east}
    \inheritanchor[from=rectangle]{south west}
    \inheritanchor[from=rectangle]{south east}

    % These have shifted now, since the three-point valve middle is not the node center anymore:
    \anchor{center}{
        \northeast \pgf@xb=\pgf@x \pgf@yb=\pgf@y
        \southwest \pgf@xa=\pgf@x \pgf@ya=\pgf@y
        \pgfmathsetlength\pgf@x{\pgf@xa + 0.5*0.75*(\pgf@xb - \pgf@xa)}
        \pgfmathsetlength\pgf@y{\pgf@ya + 0.5*(\pgf@yb - \pgf@ya)}
    }
    \anchor{north}{
        \northeast \pgf@xb=\pgf@x \pgf@yb=\pgf@y
        \southwest \pgf@xa=\pgf@x
        \pgfmathsetlength\pgf@x{\pgf@xa + 0.5*0.75*(\pgf@xb - \pgf@xa)}
        \pgfmathsetlength\pgf@y{\pgf@yb}
    }
    \anchor{south}{
        \northeast \pgf@xb=\pgf@x
        \southwest \pgf@xa=\pgf@x \pgf@ya=\pgf@y
        \pgfmathsetlength\pgf@x{\pgf@xa + 0.5*0.75*(\pgf@xb - \pgf@xa)}
        \pgfmathsetlength\pgf@y{\pgf@ya}
    }

    % The triangles meeting at their middle, each with an aspect ratio of 0.75 (base) to 0.5 (height)
    \backgroundpath{
        \southwest \pgf@xa=\pgf@x \pgf@ya=\pgf@y
        \northeast \pgf@xb=\pgf@x \pgf@yb=\pgf@y

        \pgfpathmoveto{\southwest}
        \pgfpathlineto{\pgfpoint{\pgf@xa + 0.75*(\pgf@xb - \pgf@xa)}{\pgf@yb}}
        \pgfpathlineto{\pgfpoint{\pgf@xa}{\pgf@yb}}
        \pgfpathlineto{\pgfpoint{\pgf@xa + 0.75*0.5*(\pgf@xb - \pgf@xa)}{\pgf@ya + 0.5*(\pgf@yb - \pgf@ya)}}% Valve centerpoint
        \pgfpathlineto{\pgfpoint{\pgf@xb}{\pgf@ya + 0.8*(\pgf@yb - \pgf@ya)}}
        \pgfpathlineto{\pgfpoint{\pgf@xb}{\pgf@ya + 0.2*(\pgf@yb - \pgf@ya)}}
        \pgfpathlineto{\pgfpoint{\pgf@xa + 0.75*0.5*(\pgf@xb - \pgf@xa)}{\pgf@ya + 0.5*(\pgf@yb - \pgf@ya)}}% Valve centerpoint
        \pgfpathlineto{\pgfpoint{\pgf@xa + 0.75*(\pgf@xb - \pgf@xa)}{\pgf@ya}}
        \pgfpathclose
    }
}

\pgfdeclareshape{fourway valve}{%
    \inheritsavedanchors[from=rectangle]% Inherit from rectangle, should be good enough
    \inheritanchorborder[from=rectangle]
    \inheritanchor[from=rectangle]{north}
    \inheritanchor[from=rectangle]{north west}
    \inheritanchor[from=rectangle]{north east}
    \inheritanchor[from=rectangle]{center}
    \inheritanchor[from=rectangle]{west}
    \inheritanchor[from=rectangle]{east}
    \inheritanchor[from=rectangle]{mid}
    \inheritanchor[from=rectangle]{mid west}
    \inheritanchor[from=rectangle]{mid east}
    \inheritanchor[from=rectangle]{base}
    \inheritanchor[from=rectangle]{base west}
    \inheritanchor[from=rectangle]{base east}
    \inheritanchor[from=rectangle]{south}
    \inheritanchor[from=rectangle]{south west}
    \inheritanchor[from=rectangle]{south east}

    \backgroundpath{%
        % Store lower left in xa/ya and upper right in xb/yb
        \southwest \pgf@xa=\pgf@x \pgf@ya=\pgf@y
        \northeast \pgf@xb=\pgf@x \pgf@yb=\pgf@y

        % This yields four triangles meeting in the middle, each with an aspect ratio of 0.6/1.
        % Construct main path, basically 'two triangles touching'
        \pgfpathmoveto{\pgfpoint{\pgf@xa + 0.2*(\pgf@xb - \pgf@xa)}{\pgf@ya}}
            \pgfpathlineto{\pgfpoint{\pgf@xa + 0.8*(\pgf@xb - \pgf@xa)}{\pgf@yb}}
            \pgfpathlineto{\pgfpoint{\pgf@xa + 0.2*(\pgf@xb - \pgf@xa)}{\pgf@yb}}
            \pgfpathlineto{\pgfpoint{\pgf@xa + 0.8*(\pgf@xb - \pgf@xa)}{\pgf@ya}}
        \pgfpathclose
    
        % Do the same again in the horizontal direction:
        \pgfpathmoveto{\pgfpoint{\pgf@xa}{\pgf@ya + 0.2*(\pgf@yb - \pgf@ya)}}
            \pgfpathlineto{\pgfpoint{\pgf@xa}{\pgf@ya + 0.8*(\pgf@yb - \pgf@ya)}}
            \pgfpathlineto{\pgfpoint{\pgf@xb}{\pgf@ya + 0.2*(\pgf@yb - \pgf@ya)}}
            \pgfpathlineto{\pgfpoint{\pgf@xb}{\pgf@ya + 0.8*(\pgf@yb - \pgf@ya)}}
        \pgfpathclose
    }
}

\pgfdeclareshape{controlvalve}{
    \inheritsavedanchors[from=valve]
    \inheritanchorborder[from=valve]
    \inheritanchor[from=valve]{north}
    \inheritanchor[from=valve]{north west}
    \inheritanchor[from=valve]{north east}
    \inheritanchor[from=valve]{center}
    \inheritanchor[from=valve]{west}
    \inheritanchor[from=valve]{east}
    \inheritanchor[from=valve]{mid}
    \inheritanchor[from=valve]{mid west}
    \inheritanchor[from=valve]{mid east}
    \inheritanchor[from=valve]{base}
    \inheritanchor[from=valve]{base west}
    \inheritanchor[from=valve]{base east}
    \inheritanchor[from=valve]{south}
    \inheritanchor[from=valve]{south west}
    \inheritanchor[from=valve]{south east}

    \inheritbackgroundpath[from={valve}]

    \beforebackgroundpath{%
        % Store lower left in xa/ya and upper right in xb/yb
        \southwest \pgf@xa=\pgf@x \pgf@ya=\pgf@y
        \northeast \pgf@xb=\pgf@x \pgf@yb=\pgf@y

        % Extension line to control wheel:
        \pgfpathmoveto{\pgfpoint{(\pgf@xa + \pgf@xb)/2}{(\pgf@ya + \pgf@yb)/2}}
        \pgfpathlineto{\pgfpoint{\pgf@xb}{(\pgf@ya + \pgf@yb)/2}}
    
        % The factors 0.3, 0.4 and 0.3 need to add up to 1 to give a symmetrical, nice semicircle.
        % Factor 0.4 is divided by 2 to give the radius
        \pgfpathmoveto{\pgfpoint{\pgf@xb}{\pgf@ya + 0.3*(\pgf@yb - \pgf@ya)}}
        \pgfpatharc{-90}{90}{0.4/2*(\pgf@yb - \pgf@ya)}
        \pgfpathclose
        \pgfsetfillcolor{black}
        \pgfusepath{fill, stroke}
    }
}

\pgfdeclareshape{pump}{%
    \inheritsavedanchors[from=circle]%
    \inheritanchorborder[from=circle]
    \inheritbackgroundpath[from={circle}]
    \inheritanchor[from=circle]{center}
    \inheritanchor[from=circle]{north}
    \inheritanchor[from=circle]{south}
    \inheritanchor[from=circle]{west}
    \inheritanchor[from=circle]{east}

    \beforebackgroundpath{
        % The simple-most approach:
        \pgfpathmoveto{\pgf@anchor@pump@west}
            \pgfpathlineto{\pgf@anchor@pump@north}
            \pgfpathlineto{\pgf@anchor@pump@east}
        \pgfusepath{stroke}
    }
}

\pgfdeclareshape{compressor}{%
    \inheritsavedanchors[from=circle]%
    \inheritanchorborder[from=circle]%
    \inheritbackgroundpath[from={circle}]%
    \inheritanchor[from=circle]{center}%
    \inheritanchor[from=circle]{north}%
    \inheritanchor[from=circle]{south}%
    \inheritanchor[from=circle]{west}%
    \inheritanchor[from=circle]{east}%

    % Circle has savedanchor of 'centerpoint' and saveddim of 'radius', that's it

    \beforebackgroundpath{
        \centerpoint \pgf@yc=\pgf@y

        % Pythagorean theorem is enough here.
        % Assumption is that default compressor shape 'points to the right'.
        % This is important to get coorect behaviour out of the 'sloped' node option!

        % Idea (Example for top right point):
        % 1. Go to center point.
        % 2. On y-axis, move a fraction of the radius up. This is the y-portion of our top-right coordinate point.
        % 3. On x-axis, move as far right as Pythagorean theorem dictates (hypotenuse is the radius vector, other known length is the y-length we just specified)
        % 4. Repeat process to draw and move accordingly; don't forget squares and sqrt() in correct places.
    
        \pgfmathsetmacro{\upperyfraction}{0.3}% y-coordinate of the upper points is \upperyfraction*\radius to the sides of the centerpoint
        \pgfmathsetmacro{\loweryfraction}{0.7}% y-coordinate of the lower points is \loweryfraction*\radius to the sides of the centerpoint

        \foreach \signdirection in {1, -1}{% For left and right side
            % Upper (always same x-dimension, only y switches sign):
            \pgfpathmoveto{\pgfpoint{sqrt(\radius^2 - (\pgf@yc + \upperyfraction*\radius)^2)}{\pgf@yc + \signdirection*\upperyfraction*\radius}}
            % Bottom (always same x-dimension, only y switches sign):
            \pgfpathlineto{\pgfpoint{-1*sqrt(\radius^2 - (\pgf@yc - \loweryfraction*\radius)^2)}{\pgf@yc + \signdirection*\loweryfraction*\radius}}
        }

        \pgfusepath{stroke}
    }
}

\pgfdeclareshape{heat exchanger}{
    \inheritsavedanchors[from=rectangle]% Inherit from rectangle, should be good enough. Rectangle has to 'saved anchors': southwest and northeast
    \inheritanchorborder[from=rectangle]
    % Doesn't hurt to also inherit specific anchors:
    \inheritanchor[from=rectangle]{north}
    \inheritanchor[from=rectangle]{north west}
    \inheritanchor[from=rectangle]{north east}
    \inheritanchor[from=rectangle]{center}
    \inheritanchor[from=rectangle]{west}
    \inheritanchor[from=rectangle]{east}
    \inheritanchor[from=rectangle]{mid}
    \inheritanchor[from=rectangle]{mid west}
    \inheritanchor[from=rectangle]{mid east}
    \inheritanchor[from=rectangle]{base}
    \inheritanchor[from=rectangle]{base west}
    \inheritanchor[from=rectangle]{base east}
    \inheritanchor[from=rectangle]{south}
    \inheritanchor[from=rectangle]{south west}
    \inheritanchor[from=rectangle]{south east}

    \inheritbackgroundpath[from={rectangle}]

    % Anchor to 'in' port of heat exchanger.
    % This code does not get executed necessarily; we cannot use the \pgfsetmacros from below.
    % A solution is probably \pgfkeys, but this will do for now.
    % IT NEEDS MANUAL ADJUSTING IF THE SHAPE IS CHANGED.
    \anchor{in}{%
        \southwest \pgf@xa=\pgf@x \pgf@ya=\pgf@y
        \northeast \pgf@xb=\pgf@x \pgf@yb=\pgf@y
        \pgfmathsetlength\pgf@x{\pgf@xa + (1 - 0.7)/2*(\pgf@xb - \pgf@xa)}
        \pgfmathsetlength\pgf@y{\pgf@ya - 0.3*(\pgf@yb - \pgf@ya)}
    }
    \anchor{out}{%
        \southwest \pgf@xa=\pgf@x \pgf@ya=\pgf@y
        \northeast \pgf@xb=\pgf@x \pgf@yb=\pgf@y
        \pgfmathsetlength\pgf@x{\pgf@xb - (1 - 0.7)/2*(\pgf@xb - \pgf@xa)}
        \pgfmathsetlength\pgf@y{\pgf@ya - 0.3*(\pgf@yb - \pgf@ya)}
    }

    \beforebackgroundpath{
        \southwest \pgf@xa=\pgf@x \pgf@ya=\pgf@y
        \northeast \pgf@xb=\pgf@x \pgf@yb=\pgf@y

        \pgfmathsetmacro{\heatsymbolextension}{0.3}% How far out the 'legs' extend out of the rectangle as a fraction of the overall node's height
        \pgfmathsetmacro{\heatsymbolwidth}{0.7}% The inner drawing's width as a fraction of the node's overall width
        \pgfmathsetmacro{\heatsymbolheight}{0.8}% How high the inner drawing reaches as a fraction of the node's overall height (at 1, it touches the upper rectangle border)
        \pgfmathsetmacro{\heatsymbolindentation}{0.3}% How 'indented' the triangle is, i.e. the vertical distance between the inner drawings highest point and the triangle's down-pointing 'tip'

        \pgfpathmoveto{\pgfpoint{\pgf@xa + (1 - \heatsymbolwidth)/2*(\pgf@xb - \pgf@xa)}{\pgf@ya - \heatsymbolextension*(\pgf@yb - \pgf@ya)}}% Bottom left corner
        \pgfpathlineto{\pgfpoint{\pgf@xa + (1 - \heatsymbolwidth)/2*(\pgf@xb - \pgf@xa)}{\pgf@ya + \heatsymbolheight*(\pgf@yb - \pgf@ya)}}% Straight line up
        \pgfpathlineto{\pgfpoint{\pgf@xa + 0.5*(\pgf@xb - \pgf@xa)}{\pgf@ya + (\heatsymbolheight - \heatsymbolindentation)*(\pgf@yb - \pgf@ya)}}% Line down, to the right; x-coordinate is the middle of the node
        \pgfpathlineto{\pgfpoint{\pgf@xb - (1 - \heatsymbolwidth)/2*(\pgf@xb - \pgf@xa)}{\pgf@ya + \heatsymbolheight*(\pgf@yb - \pgf@ya)}}% Line up, to the right
        \pgfpathlineto{\pgfpoint{\pgf@xb - (1 - \heatsymbolwidth)/2*(\pgf@xb - \pgf@xa)}{\pgf@ya - \heatsymbolextension*(\pgf@yb - \pgf@ya)}}% Line straight down
        \pgfusepath{stroke}
    }
}

\pgfdeclareshape{simple heat exchanger}{% Just a cross-out rectangle
    \inheritsavedanchors[from=rectangle]% Inherit from rectangle, should be good enough. Rectangle has to 'saved anchors': southwest and northeast
    \inheritanchorborder[from=rectangle]
    % Doesn't hurt to also inherit specific anchors:
    \inheritanchor[from=rectangle]{north}
    \inheritanchor[from=rectangle]{north west}
    \inheritanchor[from=rectangle]{north east}
    \inheritanchor[from=rectangle]{center}
    \inheritanchor[from=rectangle]{west}
    \inheritanchor[from=rectangle]{east}
    \inheritanchor[from=rectangle]{mid}
    \inheritanchor[from=rectangle]{mid west}
    \inheritanchor[from=rectangle]{mid east}
    \inheritanchor[from=rectangle]{base}
    \inheritanchor[from=rectangle]{base west}
    \inheritanchor[from=rectangle]{base east}
    \inheritanchor[from=rectangle]{south}
    \inheritanchor[from=rectangle]{south west}
    \inheritanchor[from=rectangle]{south east}

    \inheritbackgroundpath[from={rectangle}]

    \beforebackgroundpath{
        \southwest \pgf@xa=\pgf@x \pgf@ya=\pgf@y
        \northeast \pgf@xb=\pgf@x \pgf@yb=\pgf@y

        \pgfpathmoveto{\pgfpoint{\pgf@xa}{\pgf@yb}}% Top left corner
        \pgfpathlineto{\pgfpoint{\pgf@xb}{\pgf@ya}}% Bottom right corner
        \pgfusepath{stroke}
    }
}

\pgfkeys{/pgf/.cd,
    radiator offset x/.initial=0.3em,
    radiator offset y/.initial=0.3em
}
\pgfdeclareshape{radiator}{%
    \inheritsavedanchors[from=rectangle]% Inherit from rectangle, should be good enough. Rectangle has to 'saved anchors': southwest and northeast
    \inheritanchorborder[from=rectangle]
    % Doesn't hurt to also inherit specific anchors:
    \inheritanchor[from=rectangle]{north}
    \inheritanchor[from=rectangle]{north west}
    \inheritanchor[from=rectangle]{north east}
    \inheritanchor[from=rectangle]{center}
    \inheritanchor[from=rectangle]{west}
    \inheritanchor[from=rectangle]{east}
    \inheritanchor[from=rectangle]{mid}
    \inheritanchor[from=rectangle]{mid west}
    \inheritanchor[from=rectangle]{mid east}
    \inheritanchor[from=rectangle]{base}
    \inheritanchor[from=rectangle]{base west}
    \inheritanchor[from=rectangle]{base east}
    \inheritanchor[from=rectangle]{south}
    \inheritanchor[from=rectangle]{south west}
    \inheritanchor[from=rectangle]{south east}

    \saveddimen{\xoffset}{% https://tex.stackexchange.com/a/181283/120853
        \pgfmathsetlength\pgf@x{\pgfkeysvalueof{/pgf/radiator offset x}}
    }
    \saveddimen{\yoffset}{
        \pgfmathsetlength\pgf@y{\pgfkeysvalueof{/pgf/radiator offset y}}
    }

    % In the middle of the right surface:
    \anchor{right}{%
        \southwest \pgf@ya=\pgf@y
        \northeast \pgf@xb=\pgf@x \pgf@yb=\pgf@y
        \pgfmathsetlength\pgf@x{\pgf@xb + \xoffset/2}
        \pgfmathsetlength\pgf@y{(\pgf@ya + \pgf@yb)/2 + \yoffset/2}
    }
    % In the middle of the top surface:
    \anchor{top}{%
        \southwest \pgf@xa=\pgf@x
        \northeast \pgf@xb=\pgf@x \pgf@yb=\pgf@y
        \pgfmathsetlength\pgf@x{(\pgf@xa + \pgf@xb)/2 + \xoffset/2}
        \pgfmathsetlength\pgf@y{\pgf@yb + \yoffset/2}
    }
    % Exit at the thermostat:
    \anchor{exit}{%
        \southwest \pgf@ya=\pgf@y
        \northeast \pgf@xb=\pgf@x \pgf@yb=\pgf@y
        \pgfmathsetlength\pgf@x{\pgf@xb + \xoffset/2}
        \pgfmathsetlength\pgf@y{\pgf@ya + 0.8*(\pgf@yb - \pgf@ya) + \yoffset/2}
    }
    % Lower exit on the right side, below the thermostat:
    \anchor{lower exit}{%
        \southwest \pgf@ya=\pgf@y
        \northeast \pgf@xb=\pgf@x \pgf@yb=\pgf@y
        \pgfmathsetlength\pgf@x{\pgf@xb + \xoffset/2}
        \pgfmathsetlength\pgf@y{\pgf@ya + 0.2*(\pgf@yb - \pgf@ya) + \yoffset/2}
    }
    % Entry in the bottom left somewhere:
    \anchor{lower entry}{%
    \northeast \pgf@yb=\pgf@y
    \southwest \pgf@ya=\pgf@y% x is now given implicitly
    \pgfmathsetlength\pgf@y{\pgf@ya + 0.2*(\pgf@yb - \pgf@ya)}
    }

    % Alternative entry in the top left somewhere:
    \anchor{upper entry}{%
        \northeast \pgf@yb=\pgf@y
        \southwest \pgf@ya=\pgf@y% x is now given implicitly
        \pgfmathsetlength\pgf@y{\pgf@ya + 0.8*(\pgf@yb - \pgf@ya)}
    }

    % Front surface is inherited from rectangle:
    \inheritbackgroundpath[from={rectangle}]

    % Background is inherited already, so add behind background path:
    \behindbackgroundpath{%
        % Store lower left in xa/ya and upper right in xb/yb
        \southwest \pgf@xa=\pgf@x \pgf@ya=\pgf@y
        \northeast \pgf@xb=\pgf@x \pgf@yb=\pgf@y

        % Top surface:
        \pgfpathmoveto{\pgfpoint{\pgf@xa}{\pgf@yb}}% Top left
        \pgfpathlineto{\pgfpointadd{\pgfpoint{\pgf@xa}{\pgf@yb}}{\pgfqpoint{\xoffset}{\yoffset}}}% Top left plus offset vector
        \pgfpathlineto{\pgfpointadd{\northeast}{\pgfqpoint{\xoffset}{\yoffset}}}% Top right (northeast) plus offset vector
        \pgfpathlineto{\northeast}
        \pgfpathclose

        % Access current fillcolor as specified by user
        % Make sides darker than current fill color for some 3D effect
        \colorlet{currentfill}{\tikz@fillcolor}
        \pgfsetfillcolor{currentfill!90!black}
        \pgfusepath{fill, stroke}
    
        % Right surface:
        \pgfpathmoveto{\northeast}
        \pgfpathlineto{\pgfpointadd{\northeast}{\pgfqpoint{\xoffset}{\yoffset}}}% Top right corner plus offset vector
        \pgfpathlineto{\pgfpointadd{\pgfpoint{\pgf@xb}{\pgf@ya}}{\pgfqpoint{\xoffset}{\yoffset}}}% Bottom right corner plus offset vector
        \pgfpathlineto{\pgfpoint{\pgf@xb}{\pgf@ya}}% Bottom right corner
        \pgfpathclose
        \pgfsetfillcolor{currentfill!80!black}% Even darker than top surface
        \pgfusepath{fill, stroke}

        % Draw an ellipse to fake a thermostat or pipe exit.
        % Position it at 0.8 / 80% height
        % Ellipse axes are confusing and I did not do it properly fully
        \pgfpathellipse{\pgfpointadd{\pgfpoint{\pgf@xb}{\pgf@ya + 0.8*(\pgf@yb - \pgf@ya)}}{\pgfpointscale{0.5}{\pgfqpoint{\xoffset}{\yoffset}}}}{\pgfpoint{0.2*\xoffset}{0.2*\yoffset}}{\pgfpoint{0pt}{1pt}}
        \pgfsetfillcolor{black}
        \pgfusepath{fill}
    }
    % Background is already the inherited rectangle. Draw on top of it with beforebackgroundpath
    \beforebackgroundpath{
        % Store lower left in xa/ya and upper right in xb/yb
        \southwest \pgf@xa=\pgf@x \pgf@ya=\pgf@y
        \northeast \pgf@xb=\pgf@x \pgf@yb=\pgf@y

        % Draw vertical lines on top of front surface:
        \foreach \posfraction in {0.2, 0.4, ..., 0.8}{
            \pgfpathmoveto{\pgfpoint{\pgf@xa + \posfraction*(\pgf@xb - \pgf@xa)}{\pgf@ya + 0.2*(\pgf@yb - \pgf@ya)}}
            \pgfpathlineto{\pgfpoint{\pgf@xa + \posfraction*(\pgf@xb - \pgf@xa)}{\pgf@ya + 0.8*(\pgf@yb - \pgf@ya)}}
            \pgfusepath{stroke}
        }
    }
}

\pgfdeclareshape{vented radiator}{
    \inheritsavedanchors[from=radiator]% Inherit from rectangle, should be good enough. Rectangle has to 'saved anchors': southwest and northeast
    \inheritanchorborder[from=radiator]
    % Doesn't hurt to also inherit specific anchors:
    \inheritanchor[from=radiator]{north}
    \inheritanchor[from=radiator]{north west}
    \inheritanchor[from=radiator]{north east}
    \inheritanchor[from=radiator]{center}
    \inheritanchor[from=radiator]{west}
    \inheritanchor[from=radiator]{east}
    \inheritanchor[from=radiator]{mid}
    \inheritanchor[from=radiator]{mid west}
    \inheritanchor[from=radiator]{mid east}
    \inheritanchor[from=radiator]{base}
    \inheritanchor[from=radiator]{base west}
    \inheritanchor[from=radiator]{base east}
    \inheritanchor[from=radiator]{south}
    \inheritanchor[from=radiator]{south west}
    \inheritanchor[from=radiator]{south east}

    % Inherit our special nodes:
    \inheritanchor[from=radiator]{right}
    \inheritanchor[from=radiator]{top}
    \inheritanchor[from=radiator]{exit}
    \inheritanchor[from=radiator]{lower entry}
    \inheritanchor[from=radiator]{upper entry}

    % Inherit all stuff from base radiator:
    \inheritbackgroundpath[from={radiator}]
    \inheritbehindbackgroundpath[from={radiator}]
    \inheritbeforebackgroundpath[from={radiator}]

    % Anchor for the ventilation point.
    % I tried with \savedanchor, but it didn't work so we have to repeat the calculations for the ventilation device
    \anchor{ventilation}{%
        \southwest \pgf@xa=\pgf@x
        \northeast \pgf@yb=\pgf@y
        \pgfmathsetlength\pgf@x{\pgf@xa - 3pt}
        \pgfmathsetlength\pgf@y{\pgf@yb - 3pt}
    }

    \behindforegroundpath{
        \southwest \pgf@xa=\pgf@x
        \northeast \pgf@yb=\pgf@y

        \pgfpathmoveto{\pgfpoint{\pgf@xa}{\pgf@yb}}% Top left corner
        \pgfpatharc{90}{270}{3pt}% Arc from top-top left corner down with specified radius in last argument
        \pgfsetfillcolor{black}
        \pgfusepath{fill}
    }
}

\pgfdeclareshape{sensor}{
    \inheritsavedanchors[from=rectangle]% Inherit from rectangle, should be good enough. Rectangle has to 'saved anchors': southwest and northeast
    \inheritanchorborder[from=rectangle]
    % Doesn't hurt to also inherit specific anchors:
    \inheritanchor[from=rectangle]{north}
    \inheritanchor[from=rectangle]{north west}
    \inheritanchor[from=rectangle]{north east}
    \inheritanchor[from=rectangle]{center}
    \inheritanchor[from=rectangle]{west}
    \inheritanchor[from=rectangle]{east}
    \inheritanchor[from=rectangle]{south}
    \inheritanchor[from=rectangle]{south west}
    \inheritanchor[from=rectangle]{south east}

    \backgroundpath{
        % Store lower left in xa/ya and upper right in xb/yb
        \southwest \pgf@xa=\pgf@x \pgf@ya=\pgf@y
        \northeast \pgf@xb=\pgf@x \pgf@yb=\pgf@y

        \pgfpathmoveto{\pgfpoint{\pgf@xa + (\pgf@xb - \pgf@xa)/2}{\pgf@yb}}% Upper middle
        \pgfpathlineto{\pgfpoint{\pgf@xa + (\pgf@xb - \pgf@xa)/2}{\pgf@ya + 0.3*(\pgf@yb - \pgf@ya)}}% Down the middle, to 1/4 of the node height

        % These are percentages of the node height.
        % At each percentage/fraction of the height, draw a centered line with a width of a certain percentage (fractionlength) of the overall node width
        \foreach \fractionheight/\fractionlength in {
            0.3/1,%
            0.15/0.75,%
            0/0.5%
        }{
            % One horizontal line on this height:
            \pgfpathmoveto{\pgfpoint{\pgf@xa + (1 - \fractionlength)/2*(\pgf@xb - \pgf@xa)}{\pgf@ya + \fractionheight*(\pgf@yb - \pgf@ya)}}
            \pgfpathlineto{\pgfpoint{\pgf@xb - (1 - \fractionlength)/2*(\pgf@xb - \pgf@xa)}{\pgf@ya + \fractionheight*(\pgf@yb - \pgf@ya)}}
        }
    }
}

\pgfdeclareshape{levelindicator}{
    \inheritsavedanchors[from=rectangle]% Inherit from rectangle, should be good enough. Rectangle has to 'saved anchors': southwest and northeast
    \inheritanchorborder[from=rectangle]
    % Doesn't hurt to also inherit specific anchors:
    \inheritanchor[from=rectangle]{north}
    \inheritanchor[from=rectangle]{north west}
    \inheritanchor[from=rectangle]{north east}
    % \inheritanchor[from=rectangle]{center}
    \inheritanchor[from=rectangle]{west}
    \inheritanchor[from=rectangle]{east}
    \inheritanchor[from=rectangle]{south}
    \inheritanchor[from=rectangle]{south west}
    \inheritanchor[from=rectangle]{south east}

    % When no other anchor is specified, nodes are placed by their 'center' anchor.
    % Change it here to be the lower pointy triangle bit, as opposed to the rectangle center.
    % That way, the triangle always points onto the fluid surface if placed as a node on a drawn line representing this surface
    \anchor{center}{%
        \southwest \pgf@xa=\pgf@x
        \northeast \pgf@xb=\pgf@x \pgf@yb=\pgf@y
        % Lower pointy bit of triangle, as below:
        \pgfpoint{\pgf@xa + (\pgf@xb - \pgf@xa)/2}{\pgf@yb - sqrt(3)*((\pgf@xb - \pgf@xa)/2)}
    }

    \backgroundpath{
        % Store lower left in xa/ya and upper right in xb/yb
        \southwest \pgf@xa=\pgf@x \pgf@ya=\pgf@y
        \northeast \pgf@xb=\pgf@x \pgf@yb=\pgf@y

        \pgfpathmoveto{\pgfpoint{\pgf@xa}{\pgf@yb}}% Upper left corner
        \pgfpathlineto{\pgfpoint{\pgf@xb}{\pgf@yb}}% Upper right corner
        \pgfpathlineto{\pgfpoint{\pgf@xa + (\pgf@xb - \pgf@xa)/2}{\pgf@yb - sqrt(3)*((\pgf@xb - \pgf@xa)/2)}}% Forming an equilateral triangle
        \pgfpathclose

        % These are percentages of the node height.
        % At each percentage/fraction of the height, draw a centered line with a width of a certain percentage (fractionlength) of the overall node width
        \foreach \fractionheight/\fractionlength in {
            0.3/1,%
            0.15/0.75,%
            0/0.5%
        }{
            % One horizontal line on this height:
            \pgfpathmoveto{\pgfpoint{\pgf@xa + (1 - \fractionlength)/2*(\pgf@xb - \pgf@xa)}{\pgf@ya + \fractionheight*(\pgf@yb - \pgf@ya)}}
            \pgfpathlineto{\pgfpoint{\pgf@xb - (1 - \fractionlength)/2*(\pgf@xb - \pgf@xa)}{\pgf@ya + \fractionheight*(\pgf@yb - \pgf@ya)}}
        }
    }
}